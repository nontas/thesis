% !TEX root =  ../../thesis.tex
\section{Automatic Training of Deformable Models}
Herein, we present the prior work on the automatic construction of Deformable Models, which is the focus of Chapter~\ref{ch:automatic_training}. Due to the fact that manual annotation is a rather costly, labor-intensive and prune to human mistakes procedure, unsupervised and semi-supervised learning of models for the tasks of alignment, landmark localization, tracking and recognition has attracted considerable
attention~\cite{kokkinos2007unsupervised,jojic2006escaping,jiang2009learning,liu2009simultaneous,vetter1997bootstrapping,tong2012semi,cox2008least,baker2004automatic,cootes2004groupwise,peng2012rasl,weber2000unsupervised,frey2003learning,lankinen2011local,huang2012learning,learned2006data,zhu1996forms,winn2005locus}. In Chapter~\ref{ch:automatic_training}, we propose a method to automatically construct Deformable Models for object alignment and the most related works are~\cite{kokkinos2007unsupervised,vetter1997bootstrapping,baker2004automatic,cootes2004groupwise,peng2012rasl}. The related family of techniques, known as image congealing~\cite{liu2009simultaneous,learned2006data,huang2012learning,lankinen2011local}, uses implicit models to align a set of images as a whole, which means that both performing alignment to a new image and constructing a model is not straightforward. Our methodology differs from these works because we employ an explicit texture model which is learned through the process.

The two most closely related works to the proposed method are the automatic construction of AAMs in~\cite{baker2004automatic} and the so-called RASL (Robust Alignment by Sparse and Low-rank Decomposition) methodology in~\cite{peng2012rasl} for person-specific face alignment. There are two main differences between our framework and~\cite{baker2004automatic}. (1)~We use a predefined statistical shape model instead of trying to find both the shape and appearance models. We believe that with the current available optimization techniques, it is extremely difficult to simultaneously optimize for both the texture and shape parameters. (2)~We employ the robust component analysis of~\cite{tzimiropoulos2012subspace} for the appearance which deals with outliers. Thus, even though our method is similar in concept to~\cite{baker2004automatic}, these two differences make the problem feasible to solve. In particular, the methodology in~\cite{baker2004automatic} fails to create a generic model even in controlled recording conditions, due to extremely high dimensionality of the parameters to be found and to the sensitivity of the subspace method to outliers. This was probably one of the reasons why the authors demonstrate very limited and only person-specific experiments. Furthermore, our methodology bypasses some of the limitations of~\cite{peng2012rasl}, which requires the presence of only one low-rank subspace, hence it has been shown to work only for the case of congealing images of a single person. Finally, we argue that in order for an automatically constructed AAM methodology to be robust to both within-class and out-of-class outliers\footnote{Within-class outliers refer to outliers present in the image of an object such as occlusion. Out-of-class outliers refer to images of irrelevant objects or to background.}, which cannot be avoided in totally unsupervised settings, statistical component analysis techniques should be employed~\cite{baker2004automatic}.
