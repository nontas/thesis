% !TEX root =  ../../thesis.tex
\section{Deformable Models}\label{sec:deformable_models_review}
Deformable Models aim to solve the problem of generic object alignment in terms
of localization of landmark (fiducial) points that correspond to semantically
meaningful parts of the object. As explained in Sec.~\ref{sec:objectives},
although deformable models can be built for a variety of object classes, the
majority of ongoing research has focused on the task of facial alignment. This
is largely due to the plethora of existing databases with annotated facial
images (\eg, Labeled Face Parts in the Wild
(LFPW)~\cite{belhumeur2011localizing,sagonas2013semi}, Annotated Faces in the
Wild (AFW)~\cite{zhu2012face,sagonas2013semi},
HELEN~\cite{le2012interactive,sagonas2013semi},
IBUG~\cite{sagonas2013300,sagonas2013semi},
300W~\cite{sagonas2013300,sagonas2013semi,sagonas2016faces},
Annotated Facial Landmarks in the Wild (AFLW)~\cite{kostinger2011annotated},
MultiPIE~\cite{gross2010multi,sagonas2013semi}), most of which have in-the-wild
data. Recent large-scale challenges on facial
alignment~\cite{sagonas2013300,sagonas2013semi,sagonas2016faces} are
characteristic examples of the rapid progress being made in the field.

Currently, the most commonly-used and well-studied face alignment methods can
be separated in two major families: \emph{(i)}~\emph{generative} models that are
iteratively optimized using Gauss-Newton or Gradient Descent algorithms, and
\emph{(ii)}~\emph{discriminative} models that employ regression in a cascaded
manner. Deformable Models can also be split in two categories based on whether
they use \emph{(i)}~\emph{holistic} or \emph{(ii)}~\emph{part-based} appearance
representation. In the next sections, we review the related work of each
category separately.


%%% GENERATIVE MODELS
\subsection{Generative Deformable Models}
The most dominant algorithm of this category is, by far, the Active Appearance
Model (AAM), which is descendant of Active Contour Model~\cite{kass1988snakes}
and Active Shape Model~\cite{cootes1995active}. An AAM consists of parametric
linear models of the shape and appearance of an object. The shape model,
usually referred to as Point Distribution Model (PDM), is built by applying
Principal Component Analysis
(PCA)~\cite{wold1987principal,jolliffe2002principal} on a set of aligned
shapes. Similarly, the appearance model is built by applying PCA on a set of
shape-free appearance instances, acquired by warping the training images into a
reference shape. The use of a parametric statistical model gives rise to their
labeling as generative models. The AAM objective function involves the
minimization of the appearance reconstruction error with respect to the shape
parameters. AAMs were initially proposed
in~\cite{cootes1995active,cootes1998active,cootes2001active}, where the
optimization was performed by a single regression step between the current
image reconstruction residual and an increment to the shape parameters.
However, the authors in~\cite{matthews2004active,baker2004lucas} showed how to
linearize the AAM objective function and optimize it using the Gauss-Newton
algorithm, which was inspired by their Lucas-Kanade (LK)
algorithm~\cite{baker2004lucas,baker2003lucas} for parametric image alignment
with respect to the parameters of an affine transform.

Following this, Gauss-Newton optimization has been the modern de facto method
for optimizing AAMs. The most common choice for both LK and AAMs matching is
the Inverse Compositional (IC) image alignment
algorithm~\cite{baker2004lucas,matthews2004active}. IC is a non-linear,
Gauss-Newton optimization technique that aims to minimize the $\ell_2$ norm
between the warped image texture and a target texture. The target texture is
the static template image in the case of affine image alignment with LK and a
model texture instance in the case of non-rigid face alignment with AAMs. Since
IC is a Gauss-Newton optimization technique, the registration result is
sensitive to initialization and to appearance variation (illumination, pose,
identity, expression, occlusion, etc.) exposed in the input and the target
images~\cite{baker2003lucas}. Especially, in the case of AAMs with
intensity-based appearance representation and optimized with the Project-Out IC
algorithm~\cite{matthews2004active}, the model is incapable of adequately
generalizing in order to be robust to outliers. This is the main reason why
AAMs have been criticized of being able to perform well only in person specific
applications and not generic ones. Many approaches have been proposed to deal
with these issues and improve
efficiency~\cite{baker2004lucas,papandreou2008adaptive,amberg2009compositional,hager1998efficient,baker2001equivalence,liu2008sift,navarathna2011fourier,tzimiropoulos2013optimization,alabort2014bayesian,tzimiropoulos2014gauss,tzimiropoulos2014active,alabort2015unifying,alabort2016unified},
robustness~\cite{megret2010bidirectional,tzimiropoulos2011robust,lucey2013fourier,dowson2008mutual,gross2005generic,black1998eigentracking,alabort2014bayesian,alabort2015unifying,duong2015beyond,alabort2016unified}
and
generalization~\cite{gross2005generic,tzimiropoulos2012subspace,tzimiropoulos2012generic}.
Many of the proposed methods introduce algorithmic improvements. The authors
in~\cite{papandreou2008adaptive} propose an adaptation on the fitting matrix
and the employment of prior information to constrain the IC fitting process.
In~\cite{baker2003lucas,black1998eigentracking} the $\ell_2$ norm is replaced
by a robust error function and the optimization aims to solve a re-weighted
least squares problem with an iterative update of the weights. Moreover, the
method in~\cite{tzimiropoulos2011robust} aligns two images by maximizing their
gradient correlation coefficient.

Most of the existing AAM works utilize an intensity-based appearance, which is
not suitable to create a generic appearance model and achieve accurate image
alignment. However, the work presented in this thesis proves that this
limitation can be easily overpassed and highly accurate results can be
achieved. Specifically, in Chapter~\ref{ch:aam} we propose the employment of
highly-descriptive, \emph{dense} appearance features for both LK and AAMs.
Especially in the case of HOG~\cite{dalal2005histograms} and
SIFT~\cite{lowe1999object} AAMs, we demonstrate results on in-the-wild
databases that significantly outperform state-of-the-art methods in facial
alignment, which are discriminatively trained on much more data.

Feature-based image representation has gained extended attention for various
Computer Vision tasks such as image segmentation and object
alignment/recognition. There is ongoing research on the employment of features
for both LK~\cite{liu2008sift,tzimiropoulos2011robust,lucey2013fourier} and
AAMs~\cite{tzimiropoulos2012generic,cootes2001representing,gao2009gabor,lucey2013fourier,ge2013active,scott2003improving,kittipanya2006effect,stegmann2003multi,su2009texture,wolstenholme1999wavelet,darkner2004wedgelet,antonakos2014automatic,duong2015beyond}.
The authors in~\cite{liu2008sift} use correspondences between dense
SIFT~\cite{lowe1999object} descriptors for scene alignment and face
recognition. Various appearance representations are proposed
in~\cite{scott2003improving,kittipanya2006effect} to improve the performance of
AAMs. One of the first attempts for feature-based AAMs
is~\cite{cootes2001representing}. The authors use novel features based on the
orientations of gradients to represent edge structure within a regression
framework. Similar features are employed in~\cite{tzimiropoulos2012generic} to
create a robust similarity optimization criterion. In~\cite{stegmann2003multi},
the intensities appearance model is replaced by a mixture of grayscale
intensities, hue channel and edge magnitude.

Recently, more sophisticated multi-dimensional features are adopted for AAM
fitting. The work in~\cite{lucey2013fourier} proposes to apply the IC
optimization algorithm in the Fourier domain using the Gabor responses for LK
and AAMs. This is different than the framework proposed in this thesis, since
in our approach the optimization is carried out in the spatial domain.
In~\cite{su2009texture}, a new appearance representation is introduced for AAMs
by combining Gabor wavelet and Local Binary Pattern (LBP) descriptor. The work
in~\cite{gao2009gabor} is the closest to the proposed framework in this thesis
(Chapter~\ref{ch:aam}). The authors employ Gabor magnitude features summed over
either orientations or scales or both to build an appearance model. However,
even though the optimization is based on the IC technique and carried out in
the spatial domain, features are extracted at each iteration from the warped
image. Finally, similarly to~\cite{gao2009gabor}, the authors
in~\cite{ge2013active} model the characteristic functions of Gabor magnitude
and phase by using log-normal and Gaussian density functions respectively and
utilize the mean of the characteristics over orientations and scales. Very
recently, the authors of~\cite{duong2015beyond} proposed to replace the linear
shape and appearance models used in traditional AAM for deep shape and
appearance models based on restricted Boltzmann machines (RBM).



%%% Discriminative models
\subsection{Discriminative Deformable Models}\label{subsec:discriminative_models}
The methodologies of this category aim to learn a regression function that
regresses from the face's appearance (\eg, commonly handcrafted
features~\cite{lowe1999object,dalal2005histograms}) to the target output
variables (either the landmark coordinates or the parameters of a statistical
shape model (PDM)). Although the history behind using linear regression in
order to tackle the problem of face alignment spans back many
years~\cite{cootes2001active}, the research community had turned towards
alternative approaches due to the lack of sufficient data for training accurate
regression functions. Nevertheless, over the last few years regression-based
techniques have prevailed in the field thanks to the wealth of readily
available annotated data and powerful handcrafted
features~\cite{lowe1999object,dalal2005histograms}. It has been recently
shown~\cite{xiong2013supervised,xiong2015global} that a single regression step
is not sufficient for accurate generic alignment. On the contrary, a cascade of
regression functions is more beneficial and is in fact employed by all recent
discriminative
methodologies~\cite{dollar2010cascaded,burgos2013robust,yang2013face,yang2013sieving,xiong2014supervised,cao2014face,kazemi2014one,ren2014face,asthana2014incremental,tzimiropoulos2015project,zhu2015face,lee2015face,trigeorgis2016mnemonic}
which have proved to be highly efficient and to generalize well.

The most important work in the area of discriminative Deformable Models, which
can be applied to a big variety of problems that involve non-linear least
squares problems, is that of Supervised Descent Method
(SDM)~\cite{xiong2013supervised,xiong2015global,xiong2014supervised}. SDM was
the first work that presented cascaded regression as a general learning
framework for optimizing non-linear objective functions by learning a set of
rules from training data. In particular, the regressors at each cascade of SDM
are linear and learn average descent directions in the space of the objective
function. Note that in the original SDM formulation~\cite{xiong2013supervised},
even though the learnt descent directions are chained in a cascade, they are
only related between them by the variance remaining from the previous cascade.
Therefore, the initial cascade levels are prone to large descent steps which
may not generalize well. This was addressed in~\cite{xiong2015global} by
clustering the descent directions into cohesive groups during training. At test
time, a cluster is selected that represents the correct descent direction. For
example, for face alignment this requires an initial estimate of the shape and
the descent directions are clustered according to the head pose.

Many different discriminative Deformable Models have emerged, since the first
proposal of SDM. They can be approximately separated into two categories based
on the type of the employed regression function. The first category includes
methodologies that employ a linear
regression~\cite{xiong2013supervised,xiong2015global,xiong2014supervised,asthana2014incremental,tzimiropoulos2015project,zhu2015face}. These methods usually
employ hand-crafted features, such as HOG~\cite{dalal2005histograms} and
SIFT~\cite{lowe1999object}. The second category, which has proved to be more
efficient than the first one, includes methods that achieve regression via
boosting of weak learners such as random
ferns~\cite{cao2014face,burgos2013robust} or random
forests~\cite{kazemi2014one,ren2014face}. These techniques tend to utilize
data-driven features that are optimized directly by the
regressor~\cite{burgos2013robust,dollar2010cascaded,kazemi2014one}.
Furthermore, the authors in~\cite{asthana2014incremental} have proposed an
incremental algorithm which allows to parallelize the training of the cascade
levels. A method to combine multiple landmark hypotheses using Structured
Support Vector Machines was proposed in~\cite{yan2013learn}.
In~\cite{lee2015face}, the authors substitute linear regressors by ensembles of
linear and Gaussian processes regression trees. Finally, the authors
in~\cite{rivera2012learning} and~\cite{sun2013deep} learn a mapping from the
initial bounding box acquired by the face detector to the landmarks' locations
using Kernel Ridge Regression and Deep Convolutional Neural Network (DCNN),
respectively.



%%% Holistic vs Part-Based
\subsection{Holistic vs. Part-Based Deformable Models}
Until recently, all research efforts had mainly focused on developing
Deformable Models with holistic appearance
representation~\cite{cootes1998active,cootes2001active,matthews2004active,antonakos2015feature,alabort2014bayesian,antonakos2014hog,gross2005generic,tzimiropoulos2012generic,tzimiropoulos2014active}.
This means that the whole texture information inside the object's shape is
taken into account and usually warped into a canonical space using a non-linear
warping function (\eg,
Piecewise Affine Warp~\cite{baker2004lucas,baker2003lucas},
Thin-Plate Splines~\cite{bookstein1989principal}).

Nevertheless, mainly due to the high complexity when using a holistic
appearance representation, most recent existing methods started employing a
part-based one. This means that a local patch is extracted from the
neighborhood around each landmark. All the discriminative Deformable Models
mentioned in Sec.~\ref{subsec:discriminative_models} belong to this category,
whereas the first part-based AAM was proposed in~\cite{tzimiropoulos2014gauss}.
Additionally, among the most important part-based methodologies is the
generative model of Pictorial Structures
(PS)~\cite{fischler1973representation,felzenszwalb2005pictorial,andriluka2009pictorial},
its discriminative descendant Deformable Part Model
(DPM)~\cite{felzenszwalb2010object,zhu2012face}
and their extensions like Deformable Structures~\cite{zuffi2012pictorial}. PS
learns a patch expert for each part and models the shape of the object using
spring-like connections between parts based on a tree structure. Thus, a
different distribution is assumed for each pair of parts connected with an
edge, as opposed to the PCA shape model of an AAM that assumes a single
multivariate normal distribution for all parts. The optimization aims to find a
tree-based shape configuration for which the patch experts have a minimum cost
and is performed using a dynamic programming algorithm based on the distance
transform~\cite{felzenszwalb2000efficient,felzenszwalb2004distance}.

Among the first part-based Deformable Models is Active Shape Model (ASM),
initially proposed in~\cite{cootes1995active} and later re-utilized
in~\cite{saragih2011deformable}. The methodology in~\cite{cootes1995active}
fits ASM with an iterative search procedure that approximated local texture
responses with isotropic Gaussian estimators. The authors
in~\cite{cristinacce2006feature} proposed Constrained Local Model (CLM), one of
the most important existing Deformable Models. CLM is natural extension of ASM,
which employs a combined statistical model to generate local response maps. A
probabilistic interpretation of CLM is derived in~\cite{saragih2011deformable}
which utilizes non-parametric response maps. This is further extended with
shape priors in~\cite{saragih2011principal} and~\cite{belhumeur2013localizing}.
Moreover, the authors of~\cite{le2012interactive} use several independent PCA
priors to model the shape. The authors in~\cite{asthana2013robust} fit the CLM
using a robust cascaded regression approach. The authors
in~\cite{martins2014non,martins2014likelihood} use the efficient Regularized
Particle Filters (RPF) during fitting. Finally, the work
in~\cite{baltruvsaitis2014continuous} proposed to learn the local patch experts
using Continuous Conditional Neural Fields.
