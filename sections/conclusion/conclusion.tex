% !TEX root =  ../../thesis.tex
In this thesis, we proposed novel and robust Deformable Models that achieve
state-of-the-art performance on the task of landmark localization and
semi-automatic annotation of large databases. The presented work is focused on
the deformable object of human face, due to the fact that there are numerous
manually annotated facial databases with thousands of images. The thesis was
split in two parts.

In Part~\ref{part:generative_models}, we focused on developing powerful
generative Deformable Models that employ both holistic and part-based
appearance representations. Specifically, in Chapter~\ref{ch:aam} we showed
that the combination of LK (Gauss-Newton) optimization with highly-descriptive
dense features greatly improves the performance of holistic AAMs. We proved,
both theoretically and experimentally, that by extracting the features from the
input image once and then warping the features image has better performance and
lower computational complexity than computing features from the warped image at
each iteration. Additionally, we provided a deep and comprehensive comparison
between 10 popular feature descriptions and shed some light on the reasons why
some of them outperform the rest. Our formulation using alternating
optimization was tested on the tasks of image alignment and landmark
localization. Our results showed that holistic AAMs with dense HOG and SIFT
features achieve robust and accurate performance and manage to outperform
discriminative Deformable Models that are trained on much more visual data.
%
Moreover, in Chapter~\ref{ch:aps}, we proposed a powerful part-based generative
Deformable Model, referred to as APS, that combines the main ideas behind PS
and AAMs. We experimentally proved that modeling the part-based appearance of a
deformable object with a GMRF structure is more beneficial than readily
applying a PCA model. This is justified by the fact that PCA assumes
correlations between all variables, whereas the GMRF allows the selection of
meaningful correlations between specific parts of an object. Moreover, APS
utilize a spring-like deformation prior term that makes them robust to bad
initializations. We also presented a variant of the Gauss-Newton optimization
with fixed Jacobian and Hessian to fit the model, which is the fastest existing
algorithm of its kind and its low computational complexity is independent of
the employed graph structure for the GMRF. Our experimental results showed that
the method is very robust to bad initializations. Finally, its part-based
nature makes it suitable for various deformable object classes with complex
articulations.

In Part~\ref{part:generative_discriminative}, we took advantage of the
properties of the generative Deformable Models presented in
Part~\ref{part:generative_models} and combined them with powerful
discriminative Deformable Models to achieve state-of-the-art results in two
different tasks. In Chapter~\ref{ch:automatic_training} we proposed a novel
formulation for the task of semi-automatic annotation of large visual
databases. Taking advantage of the qualities of feature-based holistic AAMs
shown in Chapter~\ref{ch:aam}, the proposed framework iteratively trains a
generative and a discriminative holistic AAM ending up very accurate landmark
annotations. The only requirements of the method are a statistical shape model
of the deformable object and the true positive bounding boxes of the object
within the images. Our extensive experimental results proved that the
semi-automatically acquired annotations have comparable accuracy to manual
annotations. The proposed technique is the first one that demonstrates such
promising results on the task of automatic training of Deformable Models and
can easily be applied on various deformable object classes.
%
Additionally, in Chapter~\ref{ch:acr} we proposed ACR, a novel methodology that
achieves state-of-the-art performance on the task of landmark localization. The
method combines the descent directions of cascaded regression
\textcolor{red}{\st{with the gradient descent directions from} and} 
Gauss-Newton optimization. This combination allows ACR
to demonstrate robustness to challenging initializations and accuracy with
respect to fine details. We report state-of-the-art performance using the most
recent benchmark challenge, comparing against powerful methodologies some of
which are provided by industrial companies and are trained on much larger
training datasets.
