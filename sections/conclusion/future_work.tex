% !TEX root =  ../../thesis.tex
\section{Future Work}
The work proposed in this thesis can be further extended in various manners.
Specifically, one of the biggest limitations of Deformable Models is that they
are mostly applied and test on the object of human face, due to the numerous
annotated publicly available databases. However, the next step is to develop
generative Deformable Models for both articulated and non-articulated objects
that achieve state-of-the-art performance without requiring a huge amount of
training data. There exist very limited generative models that are suitable and
have been extensively tested on articulated
objects~\cite{tzimiropoulos2014gauss,antonakos2015active} (please refer to
Chapter~\ref{ch:aps}). This is because:
\begin{itemize}
  \item Articulated objects often have more complex texture space
  than non-articulated objects (e.g., the variations of the human body texture
  space are larger than the variations of the human face, due to clothes,
  severe self occlusions etc.). Hence, linear component analysis techniques may
  fail to properly describe these textures statistically.

  \item The majority of the employed generative component analysis techniques
  are based on holistic low-rank assumptions (such as PCA and its linear and
  non-linear variations). These methods are not able to capture the
  relationship between parts of articulated objects both in the appearance
  space, as well as in the deformable shape space.
\end{itemize}
These two challenges can be addressed in the following ways:
\begin{itemize}
  \item Apply recently developed deep methodologies for feature extraction,
  which can be trained in an unsupervised manner~\cite{bengio2009learning} or
  off-the-shelf trained DCNNs~\cite{sermanet2013overfeat}.

  \item Investigate the development of statistical component analysis
  techniques that combine low-rank and hierarchical/structured principles
  (e.g., introduce a part constraint PCA in order to encapsulate the
  dependencies between the object parts in terms of both texture and shape).
\end{itemize}

Additionally, there is plenty of room to propose novel methodologies for
training Deformable Models with limited or even no human supervision and
explore solutions towards the online incremental update of these models with
new training samples (lifelong learning). This refers to the task of constantly
updating generic Deformable Models with images coming from the web and
gradually turning them into instant specific models.
Chapter~\ref{ch:automatic_training} provides a very solid proof of concept that
supports the research towards this direction. Everyday thousands of
images are uploaded on the Internet. Hence, the methodologies should be able to
constantly incorporate new knowledge in an incremental fashion. To this end, it
should be investigated how various component analysis techniques (especially
the ones focused on articulated objects) could be reformulated so as to allow
incremental learning. Moreover, in order to learn Deformable Models of a
specific object instance, for example a person-specific body Deformable Model,
one can safely rely on the fact that these image samples are highly correlated.
Hence, it is reasonable to assume that the object's appearance will reside in a
low-rank subspace and incorporate extra low-rank constraints to powerful
generative frameworks.

Finally, it is really important for the research community to continue
developing challenging benchmarks and high-quality open-source implementations
of the various approaches. Given the strong and increasing impact of industrial
research due to the unlimited resources, open source knowledge is the only way
in which academic research can keep leading the constantly growing advances.
