% !TEX root =  ../../../thesis.tex
\section{Method}\label{sec:acr:method}
In the following sections, we follow the notation of
Secs.~\ref{sec:notation:shape} and~\ref{sec:notation:appearance} for the shape
and appearance models, respectively. Specifically, we employ the same shape representation
%%%%%%%%%%%%%%
\begin{equation}
  \mathbf{s} = \left[\boldsymbol{\ell}_1^{\mathsf{T}}, \boldsymbol{\ell}_2^{\mathsf{T}}, \ldots, \boldsymbol{\ell}_n^{\mathsf{T}} \right]^{\mathsf{T}} = \left[x_1, y_1, x_2, y_2, \ldots, x_n, y_n \right]^{\mathsf{T}}
\end{equation}
%%%%%%%%%%%%%%
as well as a shape model of the form of Eq.~\ref{equ:shape_model}. With some
abuse of notation, let us redefine the shape generation formulation of
Eq.~\ref{equ:shape_generation} as a function, \ie,
%%%%%%%%%%%%%%
\begin{equation}
  \mathbf{s}(\mathbf{p})=\bar{\mathbf{s}} + \mathbf{U}_s\mathbf{p}
\end{equation}
%%%%%%%%%%%%%%
where $\mathbf{p}=[p_1,p_2,\ldots,p_{n_s}]^{\mathsf{T}}$ is the $n_s\times1$
vector of \emph{shape parameters} that control the linear combination of the
eigenvectors.

Moreover, we employ a part-based appearance representation as explained in
Sec.~\ref{sec:notation:part_based}. With some abuse of notation, we redefine
Eq.~\ref{equ:part_based_appearance} as a function, \ie,
%%%%%%%%%%%%%%
\begin{equation}
  \mathbf{f}(\mathbf{s}) = \left[ \mathcal{F}(\mathbf{t}_{\boldsymbol{\ell}_1})^{\mathsf{T}}, \mathcal{F}(\mathbf{t}_{\boldsymbol{\ell}_2})^{\mathsf{T}}, \ldots,
  \mathcal{F}(\mathbf{t}_{\boldsymbol{\ell}_n})^{\mathsf{T}} \right]^{\mathsf{T}}
  \label{equ:feature_function_acr}
\end{equation}
%%%%%%%%%%%%%%
We also create an appearance model following the description of
Sec.~\ref{sec:notation:appearance_model}, which can be used to generate new
appearance vectors with the function
%%%%%%%%%%%%%%
\begin{equation}\label{equ:appearance_model}
    \mathbf{a}(\mathbf{c}) = \bar{\mathbf{a}} + \mathbf{U}_a\mathbf{c}
\end{equation}
%%%%%%%%%%%%%%
where $\mathbf{c}=[c_1,c_2,\ldots,c_{n_a}]^{\mathsf{T}}$ is the $n_a\times1$
vector of \emph{appearance parameters}. Finally, let us define
%%%%%%%%%%%%%%
\begin{equation}\label{equ:projection_matrix}
    \mathbf{P} = \mathbf{E}-\mathbf{U}_a\mathbf{U}_a^{\mathsf{T}}
\end{equation}
%%%%%%%%%%%%%%
which is the orthogonal complement of the appearance subspace $\mathbf{U}_a$,
where $\mathbf{E}$ denotes the $mn\times mn$ identity matrix. This projection
operator is used in order to project-out the appearance variance in the
following methods.

In the following sections, we present details of the discriminative
(Sec.~\ref{subsec:regression}) and generative (Sec.~\ref{subsec:generative})
models in order to formulate our unified model (Sec.~\ref{subsec:acr}).

% !TEX root =  ../../../../thesis.tex
\subsection{Cascaded Regression Discriminative Model}\label{subsec:regression}
Herein, we present a fully parametric cascaded regression model. We employ an
appearance model and learn a regression function that regresses from the
object's projected-out appearance to the parameters of a linear shape model.
Let us assume that we have a set of $N$ training images
$\{\mathbf{I}^1,\ldots,\mathbf{I}^N\}$ and their corresponding annotated shapes
$\{\mathbf{s}^1,\ldots,\mathbf{s}^N\}$. By projecting each ground-truth shape
to the shape basis $\mathbf{U}_s$, we get the set of ground-truth shape
parameters $\{\mathbf{p}^*_1,\ldots,\mathbf{p}^*_N\}$. Moreover, we aim to
learn a cascade of $K$ levels, \ie, $k=1,\ldots,K$.
During the training process of each level, we generate a set of $P$ perturbed
shape parameters $\mathbf{p}_{i,j}^k,~j=1,\ldots,P,~i=1,\ldots,N$, which are
sampled from a distribution that models the statistics of the detector employed
for initialization. By defining
%%%%%%%%%%%%%%
\begin{equation}
  \Delta\mathbf{p}_{i,j}^k=\mathbf{p}^*_i-\mathbf{p}_{i,j}^k,~j=1,\ldots,P,~i=1,\ldots,N
\end{equation}
%%%%%%%%%%%%%%
to be a set of shape parameters increments, the least-squares problem that we
aim to solve during training at each cascade level $k$ is
%%%%%%%%%%%%%%
\begin{equation}\label{equ:regression_cost}
    \argmin_{\mathbf{W}^k}\sum_{i=1}^N\sum_{j=1}^P\left\|\Delta\mathbf{p}_{i,j}^k - \mathbf{W}^k\mathbf{P}\left(\mathbf{f}_i(\mathbf{s}(\mathbf{p}_{i,j}^k)) - \bar{\mathbf{a}}\right)\right\|^2_2
\end{equation}
%%%%%%%%%%%%%%
where $\mathbf{P}$ is the projection operator defined in
Eq.~\ref{equ:projection_matrix} and $\mathbf{f}_i(\cdot)$ denotes the
vector of concatenated feature-based patches extracted from the training
image $\mathbf{I}^i$, as defined in Eq.~\ref{equ:feature_function_acr}. Note
that the bias term of the above objective function is substituted by the mean
appearance vector $\bar{\mathbf{a}}$. By denoting
%%%%%%%%%%%%%%
\begin{equation}
    \hat{\mathbf{f}}_{i,j,k} = \mathbf{P}\left(\mathbf{f}_i(\mathbf{s}(\mathbf{p}_{i,j}^k)) - \bar{\mathbf{a}}\right)
\end{equation}
%%%%%%%%%%%%%%
to be the projected-out residual, then the closed-form solution to the above
least-squares problem is given by
%%%%%%%%%%%%%%
\begin{equation}
    \mathbf{W}^k=\left(\sum_{i=1}^N\sum_{j=1}^P\Delta\mathbf{p}_{i,j}^k{\hat{\mathbf{f}}_{i,j,k}}^{\mathsf{T}}\right)\left(\sum_{i=1}^N\sum_{j=1}^P{\hat{\mathbf{f}}_{i,j,k}}{\hat{\mathbf{f}}_{i,j,k}}^{\mathsf{T}}\right)^{-1}
\end{equation}
%%%%%%%%%%%%%%
for each level of the cascade $k=1,\ldots,K$.

During testing, given the current estimate of the shape parameters
$\mathbf{p}_k)$ that was computed at cascade level $k$, we create the
feature-based image vector $\mathbf{f}(\mathbf{s}(\mathbf{p}_k))$,
subtract the mean appearance vector $\bar{\mathbf{a}}$, project-out the
appearance variation and estimate the shape parameters increment as
%%%%%%%%%%%%%%
\begin{equation}\label{equ:regression_dp}
    \Delta\mathbf{p}_k=\mathbf{W}^k\mathbf{P}\left(\mathbf{f}(\mathbf{s}(\mathbf{p}_k)) - \bar{\mathbf{a}}\right)
\end{equation}
%%%%%%%%%%%%%%
Then, the shape parameters vector is updated as
%%%%%%%%%%%%%%
\begin{equation}
    \mathbf{p}_k = \mathbf{p}_{k-1} + \Delta\mathbf{p}_{k-1}
\end{equation}
%%%%%%%%%%%%%%
where we set $\mathbf{p}_0=\mathbf{0}$ at the first iteration. The computational
complexity of Eq.~\ref{equ:regression_dp} per cascade level is
$\mathcal{O}(n_smn)$, thus the complexity per test image is
$\mathcal{O}(Kn_smn)$.

% !TEX root =  ../../../../thesis.tex
\subsection{Gauss-Newton Generative Model}\label{subsec:generative}
The optimization of an AAM aims to minimize the reconstruction error of the
input image with respect to the shape and appearance parameters, \ie,
%%%%%%%%%%%%%%
\begin{equation}\label{equ:aam_cost}
    \argmin_{\mathbf{p},\mathbf{c}}\left\lVert\mathbf{f}(\mathbf{s}(\mathbf{p})) - \bar{\mathbf{a}} - \mathbf{U}_a\mathbf{c}\right\rVert^2_2
\end{equation}
%%%%%%%%%%%%%%
where we employ the appearance model of Eq.~\ref{equ:appearance_model} and
$\mathbf{f}(\cdot)$ denotes the vectorized form of the input image as
defined in Eq.~\ref{equ:feature_function}. This cost function is commonly
optimized in an iterative manner using the Gauss-Newton algorithm. This
algorithm introduces an incremental update for the shape and appearance
parameters, \ie,, $\Delta\mathbf{p}$ and $\Delta\mathbf{c}$ respectively, and
solves the problem with respect to $\Delta\mathbf{p}$ by first linearizing using
first-order Taylor expansion around $\Delta\mathbf{p} = \mathbf{0}$.
The Gauss-Newton optimization can be performed either in a forward or in an
inverse manner, depending on whether the incremental update of the shape
parameters is applied on the image or the model, respectively. In this work,
we focus on the \emph{inverse} algorithm, however the forward case can be
derived in a similar way.

We follow the derivation of Chapter~\ref{ch:aam} that was first presented
in~\cite{papandreou2008adaptive} and later was readily employed
in~\cite{tzimiropoulos2013optimization,tzimiropoulos2014gauss}. By applying the
incremental shape parameters on the part of the model, the cost function of
Eq.~\ref{equ:aam_cost} becomes
%%%%%%%%%%%%%%
\begin{equation}
    \argmin_{\Delta\mathbf{p},\Delta\mathbf{c}}\left\|\mathbf{f}(\mathbf{s}(\mathbf{p})) - \bar{\mathbf{a}}(\Delta\mathbf{p}) - \mathbf{U}_a(\Delta\mathbf{p})(\mathbf{c}+\Delta\mathbf{c})\right\|^2_2
\end{equation}
%%%%%%%%%%%%%%
where
$\bar{\mathbf{a}}(\Delta\mathbf{p})=\bar{\mathbf{a}}(\mathbf{s}(\Delta\mathbf{p}))$ and
$\mathbf{U}_a(\Delta\mathbf{p})=\mathbf{U}_a(\mathbf{s}(\Delta\mathbf{p}))$.
Given the part-based nature of our model, the compositional update of the
parameters at each iteration is reduced to a simple
subtraction~\cite{tzimiropoulos2014gauss}, as
%%%%%%%%%%%%%%
\begin{equation}\label{equ:shape_parameters_update}
    \mathbf{p} \leftarrow \mathbf{p} - \Delta\mathbf{p}
\end{equation}
%%%%%%%%%%%%%%
By taking the first order Taylor expansion around
$\Delta\mathbf{p} = \mathbf{0}$, we arrive at
%%%%%%%%%%%%%%
\begin{equation}
    \argmin_{\Delta\mathbf{p},\Delta\mathbf{c}}\left\|\mathbf{f}(\mathbf{s}(\mathbf{p})) - \bar{\mathbf{a}} - \mathbf{U}_a(\mathbf{c}+\Delta\mathbf{c}) - \mathbf{J}_a\Delta\mathbf{p}\right\|^2_2
\end{equation}
%%%%%%%%%%%%%%
where
%%%%%%%%%%%%%%
\begin{equation}\label{equ:model_jacobian}
    \mathbf{J}_a=\mathbf{J}_{\bar{\mathbf{a}}} + \sum_{i=1}^mc_i\mathbf{J}_i
\end{equation}
%%%%%%%%%%%%%%
is the model Jacobian. This Jacobian consists of the mean appearance
Jacobian
$\mathbf{J}_{\bar{\mathbf{a}}}=\frac{\partial\bar{\mathbf{a}}}{\partial\mathbf{p}}$
and the Jacobian of each appearance eigenvector denoted as $\mathbf{J}_i,~i=1,\ldots,m$.

By employing the projection operator of Eq.~\ref{equ:projection_matrix} in order
to work on the orthogonal complement of the appearance subspace $\mathbf{U}_a$
and using the fact that $\mathbf{P}\mathbf{U}_a=\mathbf{P}^{\mathsf{T}}\mathbf{U}_a=\mathbf{0}$,
the above cost function can be expressed as
%%%%%%%%%%%%%%
\begin{equation}\label{euq:inverse_cost}
    \argmin_{\Delta\mathbf{p}}\left\|\mathbf{f}(\mathbf{s}(\mathbf{p})) - \bar{\mathbf{a}} - \mathbf{J}_a\Delta\mathbf{p}\right\|^2_{\mathbf{P}}
\end{equation}
%%%%%%%%%%%%%%
The solution to this least-squares problem is
%%%%%%%%%%%%%%
\begin{equation}\label{equ:inverse_solution}
    \Delta\mathbf{p} = \hat{\mathbf{H}}_a^{-1}\hat{\mathbf{J}}_a^{\mathsf{T}}(\mathbf{f}(\mathbf{s}(\mathbf{p})) - \bar{\mathbf{a}})
\end{equation}
%%%%%%%%%%%%%%
where
%%%%%%%%%%%%%%
\begin{equation}
    \hat{\mathbf{J}}_a = \mathbf{P}\mathbf{J}_a~\text{and}~\hat{\mathbf{H}}_a = \hat{\mathbf{J}}_a^{\mathsf{T}} \hat{\mathbf{J}}_a
\end{equation}
%%%%%%%%%%%%%%
are the projected-out Jacobian and Hessian matrices respectively. Note that even
though $\mathbf{J}_{\bar{\mathbf{a}}}$ and $\mathbf{J}_i$ can be precomputed,
the complete model Jacobian $\mathbf{J}_a$ depends on the appearance parameters
$\mathbf{c}$ and has to be recomputed at each iteration. Given the current
estimate of $\Delta\mathbf{p}$, the solution of $\mathbf{c}$ with respect to
the current estimate $\mathbf{c}_c$ can be retrieved as
%%%%%%%%%%%%%%
\begin{equation}\label{equ:appearance_parameters_solution}
    \mathbf{c} = \mathbf{c}_c + \mathbf{U}_a^{\mathsf{T}} \left(\mathbf{f}(\mathbf{s}(\mathbf{p})) - \bar{\mathbf{a}} - \mathbf{U_a}\mathbf{c}_c - \mathbf{J}_a\Delta\mathbf{p}\right)
\end{equation}
%%%%%%%%%%%%%%
Thus, the computational complexity of computing Eq.~\ref{equ:inverse_solution}
per iteration is $\mathcal{O}(n_sn_amn + n_s^2mn)$.
%$\mathcal{O}(n_sn_amn + n_s^2mn + n_s^3)$.
The authors in~\cite{tzimiropoulos2014gauss} suggest that by approximating the
projected-out Hessian matrix as
$\hat{\mathbf{H}}_a\approx\mathbf{J}_a^{\mathsf{T}}\mathbf{J}_a$, reduces the
complexity to $\mathcal{O}(n_amn + n_s^2mn)$ without any significant loss in
performance.

The inverse approach that we followed, which was first proposed
in~\cite{papandreou2008adaptive}, is different from the well-known project-out
inverse compositional method of~\cite{matthews2004active}. Specifically, in our
case, the linearization of the cost function is performed \textit{before}
projecting-out. On the contrary, the authors in~\cite{matthews2004active}
followed the approximation of \emph{projecting-out first and then linearising},
which eliminates the need to recompute the appearance subspace Jacobian.
However, the project-out method proposed by~\cite{matthews2004active} does not
generalize well and is not suitable for generic facial alignment.

Given the fact that $\mathbf{P}^{\mathsf{T}}=\mathbf{P}$ and
$\mathbf{P}^{\mathsf{T}}\mathbf{P}=\mathbf{P}$,
then the solution of Eq.~\ref{equ:inverse_solution} can be expanded as
%%%%%%%%%%%%%%
\begin{equation}\label{equ:inverse_solution_expanded}
    \Delta\mathbf{p} = (\mathbf{J}_a^{\mathsf{T}}\mathbf{P}\mathbf{J}_a)^{-1}\mathbf{J}_a^{\mathsf{T}}\mathbf{P}(\mathbf{f}(\mathbf{s}(\mathbf{p})) - \bar{\mathbf{a}})
\end{equation}
%%%%%%%%%%%%%%
Thus, it is worth mentioning that the solution of the regression-based model
in Eq.~\ref{equ:regression_dp} is equivalent to the Gauss-Newton solution of
Eq.~\ref{equ:inverse_solution} if the regression matrix has the form
%%%%%%%%%%%%%%
\begin{equation}\label{equ:equivalent_solutions}
    \mathbf{W}^k = (\mathbf{J}_a^{\mathsf{T}}\mathbf{P}\mathbf{J}_a)^{-1}\mathbf{J}_a^{\mathsf{T}}
\end{equation}
%%%%%%%%%%%%%%
which further reveals the equivalency of the two cost functions of
Eqs.~\ref{equ:regression_cost} and~\ref{euq:inverse_cost}.

% %% FORWARD
% \paragraph{Forward} In the forward case, the shape parameters update is performed in an additive fashion, i.e.
% \begin{equation}
% \mathbf{p} \leftarrow \mathbf{p} + \Delta\mathbf{p}
% \end{equation}
% By employing the projection matrix $\mathbf{P}$ (Eq.~\ref{equ:projection_matrix}), the cost function of Eq.~\ref{equ:aam_cost} takes the form
% \begin{equation}
% \argmin_{\Delta\mathbf{p}}\|\mathbf{f}(\mathbf{s}(\mathbf{p} + \Delta\mathbf{p})) - \bar{\mathbf{a}}\|^2_{\mathbf{P}}
% \label{equ:forward_cost}
% \end{equation}
% Opposite to the inverse case, in the forward case the linearisation around $\Delta\mathbf{p}=\mathbf{0}$ is performed on the part of the image as $\mathbf{f}(\mathbf{s}(\mathbf{p}+\Delta\mathbf{p}))\approx\mathbf{f}(\mathbf{s}(\mathbf{p})) + \mathbf{J}_{\mathbf{f}}\Delta\mathbf{p}$, where
% $\mathbf{J}_{\mathbf{f}}=\frac{\partial \mathbf{f}(\mathbf{s}(\mathbf{p}))}{\partial\mathbf{p}}$ is the image Jacobian. Consequently, the cost function of Eq.~\ref{equ:forward_cost} becomes
% \begin{equation}
% \argmin_{\Delta\mathbf{p}}\|\mathbf{f}(\mathbf{s}(\mathbf{p})) - \bar{\mathbf{a}} + \mathbf{J}_{\mathbf{f}}\Delta\mathbf{p}\|^2_{\mathbf{P}}
% \end{equation}
% The solution to this problem is readily given by
% \begin{equation}
% \Delta\mathbf{p} = - \hat{\mathbf{H}}_{\mathbf{f}}^{-1} \hat{\mathbf{J}}_{\mathbf{f}}^{\mathsf{T}} (\mathbf{f}(\mathbf{s}(\mathbf{p})) - \bar{\mathbf{a}})
% \label{equ:forward_solution}
% \end{equation}
% where
% \begin{equation}
% \begin{aligned}
% &\hat{\mathbf{J}}_{\mathbf{f}} = \mathbf{P}\mathbf{J}_{\mathbf{f}}\\
% &\hat{\mathbf{H}}_{\mathbf{f}} = \hat{\mathbf{J}}_{\mathbf{f}}^{\mathsf{T}} \hat{\mathbf{J}}_{\mathbf{f}}
% \end{aligned}
% \end{equation}
% are the projected-out Jacobian and Hessian matrices respectively. Note that the image Jacobian and, thus, the Hessian need to be computed at each iteration, which results in a computational cost of $\mathcal{O}()$ per iteration.

% !TEX root =  ../../../../thesis.tex
\subsection{Adaptive Cascaded Regression}\label{subsec:acr}
As previously explained, both the AAMs of Section~\ref{subsec:generative} and
traditional SDMs as in~\ref{subsec:regression} suffer from a number of
disadvantages. To address these disadvantages, we propose ACR which
combines the two previously described discriminative and generative optimization
problems into a single unified cost function. Specifically, by employing
the regression-based objective function of Eq.~\ref{equ:regression_cost}
along with the Gauss-Newton analytical solution of
Eq.~\ref{equ:inverse_solution}, the training procedure of ACR aims to minimize
%%%%%%%%%%%%%%
\begin{equation}\label{equ:acr_cost}
    \sum_{i=1}^N\sum_{j=1}^P\left\|\Delta\mathbf{p}_{i,j}^k - \left(\lambda^k\mathbf{W}^k - (1-\lambda^k) \mathbf{H}_{i,j}^{-1}\mathbf{J}_{i,j}^{\mathsf{T}}\right) \hat{\mathbf{f}}_{i,j,k}\right\|^2_2
\end{equation}
%%%%%%%%%%%%%%
with respect to $\mathbf{W}^k$, where
%%%%%%%%%%%%%%
\begin{equation}\label{equ:projected_out_residuals}
    \hat{\mathbf{f}}_i(\mathbf{s}(\mathbf{p}_{i,j}^k)) = \mathbf{P}\left(\mathbf{f}_i(\mathbf{s}(\mathbf{p}_{i,j}^k)) - \bar{\mathbf{a}}\right)
\end{equation}
%%%%%%%%%%%%%%
is the projected-out residual and $\mathbf{H}_{i,j}$ and $\mathbf{J}_{i,j}$
denote the Hessian and Jacobian matrices, respectively, of the Gauss-Newton
optimization algorithm per image $i=1,\ldots,N$ and per
perturbation $j=1,\ldots,P$.
$\lambda_k$ is a hyperparameter that controls the weighting between
the regression-based descent directions and the Gauss-Newton gradient
descent directions at each level of the cascade $k=1,\ldots,K$.
The negative sign in front of the gradient descent directions is
due to the fact that the shape parameters update within the
inverse Gauss-Newton optimization is performed with subtraction,
as shown in Eq.~\ref{equ:shape_parameters_update}.

%%%%%%%%%%%%%%%%%%%%%% T R A I N I N G %%%%%%%%%%%%%%%%%%%%%%%%%%%%%%%
\subsubsection{Training}
During training, ACR aims to learn a cascade of $K$ optimal linear regressors
given the gradient descent directions of each training image at each level.
Let us assume that we have a set of $N$ training images
$\{\mathbf{I}_1,\ldots,\mathbf{I}_N\}$ along with the corresponding ground
truth shapes $\{\mathbf{s}_1,\ldots,\mathbf{s}_N\}$. We also assume that
we have recovered the ground truth shape parameters for each training
image $\{\mathbf{p}^*_1,\ldots,\mathbf{p}^*_N\}$ by projecting the ground
truth shapes against the shape model.

\paragraph{Perturbations} Before performing the training
procedure, we generate a set of initializations per training image, so that the
regression function of each cascade level learns how to estimate the descent
directions that optimize from these initializations to the ground truth shape
parameters. Consequently, for each training image, we first align the mean
shape $\bar{\mathbf{s}}$ with the ground truth shape $\mathbf{s}^i$, project it
against the shape basis $\mathbf{U}_s$ and then generate a set of $P$ random
perturbations for the first four shape parameters that correspond to the
global similarity transform. Thus, we have a set of shape parameter vectors
$\mathbf{p}^k_{i,j},~\forall i=1,\ldots,N,~\forall j=1,\ldots,P$. Since the
random perturbations are applied on the first four parameters, the rest of
them remain zero, i.e., $\mathbf{p}^k_{i,j} = [{p_1}_{i,j}^k,{p_2}_{i,j}^k,{p_3}_{i,j}^k,{p_4}_{i,j}^k,\mathbf{0}^{\mathsf{T}}_{n_s-4\times1}]^{\mathsf{T}}$.
Moreover, the perturbations are sampled from a distribution that models
the statistics of the detector that will be used for automatic initialization at
testing time. This procedure is necessary only because we have a limited
number of training images and can be perceived as training data augmentation.
It could be avoided if we had more annotated images and a single initialization
per image using the detector would be adequate. The perturbations are performed
once at the beginning of the training procedure of ACR. The steps that are
applied at each cascade level $k=1,\ldots,K$, in order to estimate
$\mathbf{W}^k$, are the following:

%%%%%%%%%%%%%%%%%%%%%%%%%%%%%%%%%%%%%%%%%%%%%%%%%%%%%%%%%%%%%%%%%%%%%%
\paragraph{Step 1: Shape Parameters Increments} Given the
set of vectors $\mathbf{p}^k_{i,j}$, we formulate the set of shape parameters
increments vectors
$\Delta\mathbf{p}^k_{i,j}=\mathbf{p}^*_i-\mathbf{p}^k_{i,j},~\forall i=1,\ldots,N,~\forall j=1,\ldots,P$
and concatenate them in a $n_s\times NP$ matrix
%%%%%%%%%%%%%%
\begin{equation}\label{equ:DP}
    \Delta\mathbf{P}_k = \left[\Delta\mathbf{p}^k_{1,1}~\cdots~\Delta\mathbf{p}^k_{N,P}\right]
\end{equation}
%%%%%%%%%%%%%%

%%%%%%%%%%%%%%%%%%%%%%%%%%%%%%%%%%%%%%%%%%%%%%%%%%%%%%%%%%%%%%%%%%%%%%
\paragraph{Step 2: Projected-Out Residuals} The next step is
to compute the part-based appearance vectors from the perturbed shape locations
$\mathbf{f}_i(\mathbf{s}(\mathbf{p}^k_{i,j}))$ and then the projected-out
residuals of Eq.~\ref{equ:projected_out_residuals} $\forall i=1,\ldots,N,~\forall j=1,\ldots,P$.
These vectors are then concatenated in a single $mn\times NP$ matrix as
%%%%%%%%%%%%%%
\begin{equation}\label{equ:PHI}
    \hat{\mathbf{F}}_k = \left[\hat{\mathbf{f}}_1(\mathbf{s}(\mathbf{p}_{1,1}^k))~\cdots~\hat{\mathbf{f}}_N(\mathbf{s}(\mathbf{p}_{N,P}^k))\right]
\end{equation}
%%%%%%%%%%%%%%

%%%%%%%%%%%%%%%%%%%%%%%%%%%%%%%%%%%%%%%%%%%%%%%%%%%%%%%%%%%%%%%%%%%%%%
\paragraph{Step 3: Gradient Descent Directions} Compute the
Gauss-Newton solutions for all the images and their perturbed shapes and
concatenate them in a $n_s\times NP$ matrix as
%%%%%%%%%%%%%%
\begin{equation}
    \mathbf{G}_k = (1-\lambda^k)\left[
        \begin{array}{c}
            [\mathbf{H}^{-1}_{1,1}\mathbf{J}^{\mathsf{T}}_{1,1}\hat{\mathbf{f}}_1(\mathbf{s}(\mathbf{p}_{1,1}^k))]^{\mathsf{T}}\\
            \vdots\\

            [\mathbf{H}^{-1}_{i,j}\mathbf{J}^{\mathsf{T}}_{i,j}\hat{\mathbf{f}}_i(\mathbf{s}(\mathbf{p}_{i,j}^k))]^{\mathsf{T}}\\

            \vdots\\

            [\mathbf{H}^{-1}_{N,P}\mathbf{J}^{\mathsf{T}}_{N,P}\hat{\mathbf{f}}_N(\mathbf{s}(\mathbf{p}_{N,P}^k))]^{\mathsf{T}}
        \end{array}
    \right]^{\mathsf{T}}
    \label{equ:G}
\end{equation}
%%%%%%%%%%%%%%
Based on the expanded solution of Eq.~\ref{equ:inverse_solution_expanded},
the calculation of the Jacobian and Hessian per image involves the estimation
of the appearance parameters using Eq.~\ref{equ:appearance_parameters_solution}
and then
%%%%%%%%%%%%%%
\begin{equation}\label{equ:final_hessian_jacobian}
    \begin{aligned}
        &\mathbf{J}_{i,j} = \mathbf{J}_a\\
        &\mathbf{H}_{i,j} = \mathbf{J}_{i,j}^{\mathsf{T}} \mathbf{P} \mathbf{J}_{i,j}
    \end{aligned}
\end{equation}
%%%%%%%%%%%%%%
where $\mathbf{J}_a$ is computed based on Eq.~\ref{equ:model_jacobian} for each image.

%%%%%%%%%%%%%%%%%%%%%%%%%%%%%%%%%%%%%%%%%%%%%%%%%%%%%%%%%%%%%%%%%%%%%%
\paragraph{Step 4: Regression Descent Directions}
By using the matrices definitions of Eqs.~\ref{equ:DP},~\ref{equ:PHI} and~\ref{equ:G},
the cost function of ACR in Eq.~\ref{equ:acr_cost} takes the form
%%%%%%%%%%%%%%
\begin{equation}\label{equ:acr_cost_matrices}
\argmin_{\mathbf{W}^k}\left\|\Delta\mathbf{P}_k - \lambda^k\mathbf{W}^k\hat{\mathbf{F}}_k + \mathbf{G}_k\right\|^2_2
\end{equation}
%%%%%%%%%%%%%%
The closed-form solution of the above least-squares problem is
%%%%%%%%%%%%%%
\begin{equation}\label{equ:final_regression_solution}
\mathbf{W}^k = \frac{1}{\lambda^k} \left(\Delta\mathbf{P}_k + \mathbf{G}_k\right) \left({\hat{\mathbf{F}}_k}^{\mathsf{T}} \hat{\mathbf{F}}_k \right)^{-1} {\hat{\mathbf{F}}_k}^{\mathsf{T}}
\end{equation}
%%%%%%%%%%%%%%
Note that the regression matrix of this step is estimated only in case
$\lambda_k \geq 0$. If $\lambda_k = 0$, then we directly
set $\mathbf{W}_k=\mathbf{0}_{n_s\times mn}$

%%%%%%%%%%%%%%%%%%%%%%%%%%%%%%%%%%%%%%%%%%%%%%%%%%%%%%%%%%%%%%%%%%%%%%
\paragraph{Step 5: Shape Parameters Update}
The final step is to generate the new estimates of the shape parameters per
training image. By employing Eqs.~\ref{equ:final_regression_solution}
and~\ref{equ:final_hessian_jacobian}, this is achieved as
%%%%%%%%%%%%%%
\begin{equation}
    \mathbf{p}^{k+1}_{i,j} = \mathbf{p}^k_{i,j} + \left(\lambda_k\mathbf{W}^k - (1-\lambda_k)\mathbf{H}^{-1}_{i,j}\mathbf{J}^{\mathsf{T}}_{i,j}\right)\mathbf{f}_i(\mathbf{s}(\mathbf{p}_{i,j}^k))
\end{equation}
%%%%%%%%%%%%%%
$\forall i=1,\ldots,N$ and $\forall j=1,\ldots,P$. After obtaining
$\mathbf{p}_{i,j}^{k+1}$, steps 1-5 are repeated for the next cascade level.
%%%%%%%%%%%%%%%%%%%%%%%%%
%%%%% F I T T I N G %%%%%
%%%%%%%%%%%%%%%%%%%%%%%%%
\subsubsection{Fitting}
In the fitting phase, given an unseen testing image $\mathbf{I}$ and its initial
shape parameters $\mathbf{p}^0=[p_1^0,p_2^0,p_3^0,p_4^0,\mathbf{0}]^{\mathsf{T}}$, we
compute the parameters update at each cascade level $k=1,\ldots,K$ as
%%%%%%%%%%%%%%
\begin{equation}
    \mathbf{p}^k = \mathbf{p}^{k-1} + \left(\lambda_k\mathbf{W}^k - (1-\lambda_k)\mathbf{H}^{-1}\mathbf{J}^{\mathsf{T}}\right)\mathbf{f}(\mathbf{s}(\mathbf{p}^{k-1}))
\end{equation}
%%%%%%%%%%%%%%
where the Jacobian and Hessian are computed as described in Step 3 of the
training procedure (Eq.~\ref{equ:final_hessian_jacobian}). The computational
complexity per iteration is $\mathcal{O}(n_smn(n_a+n_s+1))$.

