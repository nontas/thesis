% !TEX root =  ../../../thesis.tex
\section{Method}
\label{sec:automatic_training:overview}
Assuming the existence of a statistical shape model of an object (PDM), our
method automatically trains a generative AAM and in extension a discriminative
AAM, by only using a dataset of totally unconstrained in-the-wild images
containing the object and the corresponding bounding boxes. This is achieved by
alternatingly constructing a generative and a discriminative Deformable Model.
At each iteration, the training of each of the two models utilizes the fitted
shapes computed with the other already trained model. This iterative procedure
is demonstrated in Fig.~\ref{fig:systemOverview}.

Specifically, we separate our set of images and the corresponding bounding
boxes in two disjoint equally-sized datasets, referred to as the
\emph{generative} and the \emph{discriminative} that are used for the training
of the respective models. The first generative model is trained on the initial
shapes extracted by initializing the PDM mean shape in the bounding boxes. At
each iteration, the currently trained generative model is used to find the
fitted shapes on the discriminative database's images. Then, a discriminative
model is trained on these shapes. At the next iteration, the currently trained
discriminative model is applied on the images of the generative database to
extract the shapes estimations. A new version of the generative model is then
trained based on these extracted shapes of the generative dataset. At the end
of this iterative procedure, we train a final generative and discriminative AAM
on the unified database of both datasets.

%
\begin{figure}[!t]
  \centering
  \includegraphics[width=0.65\linewidth]{figures/automatic_training/Overview/SystemOverview.png}
  \caption{Automatic construction of deformable models. Given two sets of disjoint in-the-wild images and the object detector bounding boxes, our method automatically trains an AAM by training a generative and a discriminative model in an alternating manner.}
  \label{fig:systemOverview}
\end{figure}
%

This alternating training of each model followed by the supply of updated
shapes to the other and vice versa manages to continuously improve the fitted
shapes, leading to more accurate models. The role of the discriminative model
is especially crucial, as it moves the generative model from the local optimum
that it stuck. Next, in
Sec.~\ref{sec:automatic_training:generative}
and~\ref{sec:automatic_training:discriminative} we present the generative and
discriminative models, respectively.
