% !TEX root =  ../../../thesis.tex
\section{Motivation}\label{sec:aam:motivation}
As explained in Sec.~\ref{sec:deformable_models_review}, the Lucas-Kanade (LK)
algorithm~\cite{lucas1981iterative,baker2004lucas} is the most important method
for the problem of aligning a given image with a template image. The method's
aim is to find the parameter values of a parametric motion model (commonly an
affine transform) that minimize the discrepancies between the two images.
Active Appearance Models (AAMs)~\cite{cootes2001active} are among the most
popular models for the task of face fitting. They are generative Deformable
Models of shape and appearance variation. Among the most efficient techniques
to optimize AAMs is Gauss-Newton, which recovers the parametric description of
a face instance. Gauss-Newton optimization for AAMs is similar to the LK
algorithm, with the difference that the registration is obtained between the
input image and a parametric appearance model instead of a static template.

The most common choice for both LK and AAMs fitting is the Inverse
Compositional (IC) image alignment
algorithm~\cite{baker2004lucas,matthews2004active}. IC is a non-linear,
Gauss-Newton optimization technique that aims to minimize the $\ell_2$ norm
between the warped image texture and a target texture. The target texture is
the static template image in the case of LK and a model texture instance in the
case of AAMs.

Since IC is a Gauss-Newton method, the registration result is sensitive to
initialization and to large appearance variations in terms of illumination,
expressions, occlusion, identity, etc. exposed in the input and the target
images~\cite{baker2003lucas}. Especially, in the case of intensities-based AAMs
with the Project-Out IC algorithm~\cite{matthews2004active}, the model is
incapable of adequately generalizing in order to be robust to outliers. This is
the main reason why AAMs have been criticized of not being adequate for generic
alignment applications and only being capable of performing well under person
specific scenarios.

In this chapter, we propose the employment of highly-descriptive, \emph{dense}
appearance features for both LK and holistic AAMs. We show that even though the
employment of dense features increases the data dimensionality, there is a
small raise in the time complexity and a significant improvement in the
alignment accuracy. We show that within the IC optimization, there is no need
to compute the dense features at each iteration from the warped image. On the
contrary, we extract the dense features from the original image once and then
warp the resulting multi-channel image at each iteration. This strategy gives
better results, as shown in our motivating experiment of
Sec.~\ref{subsec:aam:faceAlignment:comparison} and has smaller computational
complexity, as explained in Sec.~\ref{sec:aam:featureBasedOptimization} and
Tab.~\ref{tab:times}. Motivated by this observation, we present very accurate
and robust experimental results for both face alignment and fitting with
feature-based LK and AAMs, that prove their invariance to illumination and
expression changes and their generalization ability to unseen faces.

We apply the above concept for both LK and holistic AAMs by using a great
variety of widely-used features, such as Histograms of Oriented Gradients
(HOG)~\cite{dalal2005histograms}, Scale-Invariant Feature Transform
(SIFT)~\cite{lowe1999object}, Image Gradient Orientation kernel
(IGO)~\cite{tzimiropoulos2012subspace,tzimiropoulos2011robust}, Edge Structure
(ES)~\cite{cootes2001representing}, Local Binary Patterns
(LBP)~\cite{ojala1996comparative,ojala2001generalized,ojala2002multiresolution}
with variations~\cite{wolf2008descriptor}, and Gabor
filters~\cite{kovesi1997symmetry,kovesi1999image,lee1996image}. We extensively
evaluate the performance and behavior of the proposed framework on the commonly
used Yale B Database~\cite{georghiades2001from} for LK and on multiple
in-the-wild databases (LFPW~\cite{belhumeur2011localizing},
AFW~\cite{zhu2012face}, HELEN~\cite{le2012interactive},
iBUG~\cite{sagonas2013300}) for AAMs. Finally, we compare with two
state-of-the-art discriminative Deformable
Models~\cite{xiong2013supervised,asthana2013robust} and report more accurate
results.

To summarize, the contributions of this work are:
%%%%%%%%%%%%%%%
\begin{itemize}
  \item We propose the incorporation of densely-sampled, highly-descriptive
  features in the IC gradient descent framework. We show that the combination
  of \emph{(i)}~non-linear least-squares optimization with \emph{(ii)}~robust
  features (\eg, HOG, SIFT) and \emph{(iii)}~generative models can achieve
  excellent performance for the task of face alignment.

  \item We elaborate on the reasons why it is preferable to warp the features
  image at each iteration, rather than extracting features at each iteration
  from the warped image, as it is done in the relevant bibliography.

  \item Our extended experimental results provide solid comparisons between
  some of the most successful and widely-used features that exist in the
  current bibliography for the tasks of interest, by thoroughly investigating
  the features' accuracy, robustness, and speed of convergence.

  \item Our proposed HOG and SIFT holistic AAMs outperform state-of-the-art
  face fitting methods on a series of cross-database challenging in-the-wild
  experiments.
\end{itemize}
%%%%%%%%%%%%%%%

\noindent The content of this chapter is based on the following publications:
%%%%%%%%%%%%%%%
\begin{itemize}
  \item \textbf{E. Antonakos}, J. Alabort-i-Medina, G. Tzimiropoulos, and S.
  Zafeiriou.
  ``HOG Active Appearance Models'',
  \emph{Proceedings of IEEE International Conference on Image Processing (ICIP)},
  Paris, France, pp. 224-228, October 2014.

  \item \textbf{E. Antonakos}, J. Alabort-i-Medina, G. Tzimiropoulos, and S.
  Zafeiriou.
  ``Feature-Based Lucas-Kanade and Active Appearance Models'',
  \emph{IEEE Transactions on Image Processing (T-IP)},
  24(9): pp. 2617-2632, September 2015.
\end{itemize}
%%%%%%%%%%%%%%%

The rest of the chapter is structured as follows: Section~\ref{sec:aam:features}
briefly describes the used features. Section~\ref{sec:aam:inverseCompositional}
elaborates on the intensity-based IC algorithm for LK and AAMs.
Section~\ref{sec:aam:featureBasedOptimization} explains the strategy to combine
the IC optimization with dense features. Finally,
Section~\ref{sec:aam:experiments} presents extended experiments for LK and AAMs
and Section~\ref{sec:aam:conclusions} draws the conclusions.
