% !TEX root =  ../../../thesis.tex
\section{Conclusions}\label{sec:aam:conclusions}
In this chapter, we presented a novel formulation of LK and holistic AAMs
alignment algorithms which employs dense feature descriptors for the appearance
representation. We showed, both theoretically and experimentally, that by
extracting the features from the input image once and then warping the features
image has better performance and lower computational complexity than computing
features from the warped image at each iteration. This allows us to take
advantage of the descriptive qualities of various features in order to achieve
robust and accurate performance for the problems of face alignment and
fitting. Our LK experiments prove that feature-based face alignment is
invariant to person ID and extreme lighting variations. Our face fitting
experiments on challenging in-the-wild databases show that the feature-based
AAMs have the ability to generalize well to unseen faces and demonstrate
invariance to expression, pose and lighting variations. The presented
experiments also provide a comparison between various features and prove that
HOG and SIFT are the most powerful. Finally, we report face fitting results
using AAMs with HOG and SIFT features that outperform discriminative
state-of-the-art methods trained on thousands of images. We believe that the
experimental results are among the major contributions of this work, as they
emphasize that the combination of highly-descriptive features with efficient
optimization techniques leads to deformable models with remarkable performance.
