% !TEX root =  ../../../thesis.tex
\section{Feature-Based Optimization}\label{sec:aam:featureBasedOptimization}
In this section we describe the combination of the IC algorithm with the
feature-based appearance of Eq.~\ref{equ:featuresFunction}. The keypoint of this
combination is that there are two different ways of conducting the composition
of the features function $\mathcal{F}$ and the warp function $\mathcal{W}$ on an image.
Given an image $\mathbf{t}$ and the warp parameters $\mathbf{p}$, the warped
feature-based image $\mathbf{f}$ can be obtained with the two following composition
directions:
\begin{itemize}
\item[$\bullet$] \emph{Features from warped image:}
\begin{equation}
\mathbf{f}=\mathcal{F}\left(\mathbf{t}(\mathcal{W}(\mathbf{p}))\right)
\label{equ:warpThenFeatures}
\end{equation}
\item[$\bullet$] \emph{Warping on features image:}
\begin{equation}
\mathbf{f}=\mathbf{t}_{\mathcal{F}}(\mathcal{W}(\mathbf{p}))\text{~where~}\mathbf{t}_{\mathcal{F}}=\mathcal{F}(\mathbf{t})
\label{equ:featuresThenWarp}
\end{equation}
\end{itemize}
The composition order of these two cases is shown in Fig.~\ref{fig:warpFeaturesCases}.
In the following subsections we present the incorporation of these two functions
compositions in the IC algorithm and explain why the second one is preferable.
For simplicity, we use the LK-IC algorithm (Sec.~\ref{subsec:aam:inverseCompositional:LK-IC}) for
face alignment that does not include appearance variation.

% Composition Directions
\begin{figure}[!h]
\centering
\subfloat[Features from warped image.]{\includegraphics[width=0.8\linewidth]{figures/feature_based_aam/2_CompositionDirections/warpFeatures}
\label{fig_first_case}}
\hfil
\subfloat[Warping on features image.]{\includegraphics[width=0.8\linewidth]{figures/feature_based_aam/2_CompositionDirections/featuresWarp}
\label{fig_second_case}}
\caption{The two possible composition directions of the feature extraction function $\mathcal{F}$ and the warp function $\mathcal{W}(\mathbf{p})$.}
\label{fig:warpFeaturesCases}
\end{figure}
%

% Features function and warp function computational complexity
\subsection{Warp Function Computational Complexity}
As shown in Sec.~\ref{sec:aam:features:complexity}, the computational cost of
the feature extraction function $\mathcal{F}(\mathbf{t})$ is
$\mathcal{O}(L_TD^3)$, where $L_T=HW$ is the resolution of the image
$\mathbf{t}$. Regarding the warp function, we need to consider that the warping
of a $D$-channel image, $\mathbf{t}(\mathcal{W}(\mathbf{p}))$, includes the
three following steps:
%%%%%%%%%%%%%%%%%
\begin{enumerate}
  \item Synthesis of the shape model instance $\mathbf{s}$, generated as in
  Eq.~\ref{equ:shape_generation} using the weights $\mathbf{p}$,
  which has a cost of $\mathcal{O}(2n_sn)$.

  \item Computation of the mapping of each pixel in the mean shape
  $\bar{\mathbf{s}}$ to the synthesized shape instance. This firstly involves
  the triangulation of the shape instance in $N_{tr}$ number of triangles (same
  as the number of triangles of the mean shape) using Delaunay
  triangulation~\cite{lee1980two}. Then, six affine transformation parameters
  are computed for each triangle based on the coordinates of the corresponding
  triangles' vertexes. Finally, the transformed location of each point within
  each triangle is evaluated. Thus, the complexity of this step is
  $\mathcal{O}(6N_{tr}\frac{m}{N_{tr}})=\mathcal{O}(6m)$.

  \item Copying the values of all channels $D$ for all pixels from the input
  image to the reference frame $\bar{\mathbf{s}}$ ($\mathcal{O}(Dm)$).
\end{enumerate}
%%%%%%%%%%%%%%%
Consequently, taking into account that $(6+D)m\gg2n_sn$, the overall
computational complexity of warping a multi-channel image is
$\mathcal{O}((6+D)m)$.

% Features from warped image
\subsection{Optimization with Features from Warped Image}
From Eqs.~\ref{equ:LKICcost}~and~\ref{equ:warpThenFeatures} we get the cost
function of minimizing
%%%%%%%%%%%%%%
\begin{equation}
  \argmin_{\Delta\mathbf{p}} \left\lVert \mathcal{F}(\mathbf{t}(\mathcal{W}(\mathbf{p}))) - \mathcal{F}(\bar{\mathbf{a}}(\mathcal{W}(\Delta\mathbf{p}))) \right\rVert^2
\end{equation}
%%%%%%%%%%%%%%
with respect to $\Delta\mathbf{p}$. Thus, the first-order Taylor expansion of
this expression around $\Delta\mathbf{p}=\mathbf{0}$
is
%%%%%%%%%%%%%%
\begin{equation}
  \mathcal{F}(\bar{\mathbf{a}}(\mathcal{W}(\Delta\mathbf{p}))) \approx \mathcal{F}(\bar{\mathbf{a}}) + \frac{\partial\mathcal{F}}{\partial\bar{\mathbf{a}}} \nabla\bar{\mathbf{a}} \left.\frac{\partial\mathcal{W}}{\partial\mathbf{p}}\right|_{\mathbf{p}=\mathbf{0}}\Delta\mathbf{p}
\end{equation}
%%%%%%%%%%%%%%
Since it is not possible to compute
$\frac{\partial\mathcal{F}}{\partial\bar{\mathbf{a}}}$,
we make the approximation
$\frac{\partial\mathcal{F}}{\partial\bar{\mathbf{a}}}\nabla\bar{\mathbf{a}}\approx\nabla\mathcal{F}(\bar{\mathbf{a}})$
and the linearization becomes
%%%%%%%%%%%%%%
\begin{equation}
  \mathcal{F}(\bar{\mathbf{a}}(\mathcal{W}(\Delta\mathbf{p})))\approx\mathcal{F}(\bar{\mathbf{a}})+\nabla\mathcal{F}(\bar{\mathbf{a}})\left.\frac{\partial\mathcal{W}}{\partial\mathbf{p}}\right|_{\mathbf{p}=\mathbf{0}}\Delta\mathbf{p}
  \label{equ:featureLinearization}
\end{equation}
%%%%%%%%%%%%%%
Consequently, in every IC repetition step, the warping is performed on the intensities
image ($D=1$) with the current parameters estimate ($\mathcal{O}(7m)$) and is
followed by the feature extraction ($\mathcal{O}(mD^3)$), ending up to a cost
of $\mathcal{O}(m(7+D^3))$ per iteration. Hence, by applying $k$ iterations of
the algorithm and given that $D^3\gg7$, the overall complexity of warping and
features extraction is
\begin{equation}
\mathcal{O}(kmD^3)
\label{equ:warpFeaturesCost}
\end{equation}
Note that this is only a part of the final cost, as the IC algorithm complexity
also needs to be taken into account. Moreover, in the AAMs case, it is difficult
to extract window-based features (\eg, HOG, SIFT, LBP) from the mean shape
template image, as required from the above procedure. This is because, we have
to pad the warped texture in order to compute features on the boundary, which
requires extra triangulation points.

% Warping on features image
\subsection{Optimization with Warping on Features Image}
\label{sec:aam:featuresThenWarp}
The combination of Eqs.~\ref{equ:LKICcost}~and~\ref{equ:featuresThenWarp} gives
the cost function
%%%%%%%%%%%%%%
\begin{equation}
  \argmin_{\Delta\mathbf{p}} \left\lVert \mathbf{t}_{\mathcal{F}}(\mathcal{W}(\mathbf{p})) - \bar{\mathbf{a}}_{\mathcal{F}}(\mathcal{W}(\Delta\mathbf{p})) \right\rVert^2
\end{equation}
%%%%%%%%%%%%%%
where
$\mathbf{t}_{\mathcal{F}}=\mathcal{F}(\mathbf{t})$ and
$\bar{\mathbf{a}}_{\mathcal{F}}=\mathcal{F}(\bar{\mathbf{a}})$
are the multi-channel feature-based representations of the input and the
template images respectively. The linearization around
$\Delta\mathbf{p}=\mathbf{0}$ has the same form as in
Eq.~\ref{equ:featureLinearization} of the previous case. However, in contrast
with the previous case, the warping is performed on the feature-based image.
This means that the feature extraction is performed \emph{once} on the input
image and the resulting multi-channel image is warped during each iteration.
Hence, the computational complexity of feature extraction and warping is
$\mathcal{O}((6+D)m)$ per iteration and $\mathcal{O}(k(6+D)m+L_TD^3)$
overall per image for $k$ iterations, where $L_T$ is the resolution of the input
image.

The above cost greatly depends on the input image dimensions $L_T$. In order to
override this dependency, we firstly resize the input image with respect to the
scaling factor between the face detection bounding box and the mean shape
resolution. Then, we crop the resized image in a region slightly bigger than
the bounding box. Thus, the resulting input image has resolution approximately
equal to the mean shape resolution $m$, which leads to an overall complexity
of
%%%%%%%%%%%%%%
\begin{equation}
  \mathcal{O}(km(6+D)+mD^3)
  \label{equ:featuresWarpCost}
\end{equation}
%%%%%%%%%%%%%%
for $k$ iterations. Another reason for resizing the input image is to have
correspondence on the scales on which the features are extracted, so that they
describe the same neighborhood.

The computational complexities of Eqs.~\ref{equ:warpFeaturesCost} and
\ref{equ:featuresWarpCost} are approximately equal for small number of channels
$D$ (e.g. for ES and IGO). However, this technique of warping the features image
has much smaller complexity for large values of $D$ (\eg, HOG, SIFT, LBP,
Gabor). This is because $k(D+6)<D^3$ for large values of $D$, so $km(6+D)$
can be eliminated in Eq.~\ref{equ:featuresWarpCost}. Consequently, since
$kmD^3\gg mD$, it is more advantageous to compute the features image once
and then warp the multi-channel image at each iteration. In the experiments
(Sec.~\ref{sec:aam:experiments}), we report the timings that prove the above
conclusion. Finally, we carried out an extensive experiment comparing the two
methods for face alignment (LK) in
Sec.~\ref{subsec:aam:faceAlignment:comparison}
(Fig.~\ref{fig:warpFeaturesFeaturesWarpComparison}).
The results indicate that warping the multi-channel features image performs
better, which is an additional reason to choose this composition direction
apart from the computational complexity.
