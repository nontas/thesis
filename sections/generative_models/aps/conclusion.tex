% !TEX root =  ../../../thesis.tex
\section{Conclusions}\label{sec:aps:con}
In this chapter, we proposed a powerful part-based generative model that
combines the main ideas behind PS and AAMs. APS employ a graph-based modeling
of the appearance and use a variant of the Gauss-Newton technique to optimize
with respect to the parameters of a statistical shape model.
Our experiments show 
that modeling the patch-based appearance of an object with a GMRF structure is more beneficial than applying a PCA model.
APS also introduce a spring-like deformation prior term that makes
them robust to bad initializations. The method has a close to real-time fitting
performance, which is the same independent of the graph structure that is
employed, and as shown in our experiments needs only a few iterations to
converge. Even though we show experiments only for the task of face alignment,
we believe that the method is also suitable for other object classes,
especially articulated objects (\eg, hands, body pose) for which the
combination of patch-based appearance with the deformation prior can make a
significant difference.
