% !TEX root =  ../thesis.tex
\renewcommand{\abstractname}{Abstract}
\begin{abstract}
During the last few years, we have witnessed tremendous advances in the field
of 2D Deformable Models for the problem of landmark localization. These
advances, which are mainly reported on the task of face alignment, have created
two major and opposing families of methodologies. On the one hand, there are
the generative Deformable Models that utilize a Newton-type optimization. This
family of techniques has attracted extensive research effort during the last
two decades, but has lately been criticized of achieving inaccurate
performance. On the other hand, there is the currently predominant family of
discriminative Deformable Models that treat the problem of landmark
localization as a regression problem. These techniques commonly employ cascaded
linear regression and have proved to be very accurate.

In this thesis, we argue that even though generative Deformable Models are less
accurate than discriminative, they are still very valuable for several tasks.
In the first part of the thesis, we propose two novel generative Deformable
Models. In the second part of the thesis, we show that the combination of
generative and discriminative Deformable Models achieves state-of-the-art
results on the tasks of \emph{(i)}~landmark localization and
\emph{(ii)}~semi-supervised annotation of large visual data.
\end{abstract}
