% !TEX root =  ../../thesis.tex
\section{Objectives}\label{sec:objectives}
The aim of this Ph.D. thesis is to investigate ways to address the
aforementioned challenges by combining the main concepts and advantages of generative and discriminative Deformable Models. Specifically, this work has
the following objectives:
%%%%%%%%%%%%%%%
\begin{itemize}
  \item \textbf{Objective 1: Develop generative Deformable Models that
  achieve accurate performance without requiring a large amount of training
  data.} Generative Deformable Models have attracted extended research interest
  during the last two decades. However, they have often been
  criticized~\cite{gross2005generic,van2010capturing} for their inability to
  generalize well to conditions beyond the ones exhibited in the training set
  and have been characterized as inappropriate for fitting in-the-wild images.
  As a matter of fact, they have always been regarded as ideal options to be
  used with data captured under controlled recording scenarios and for building
  instance-specific models. One of the objectives of this thesis is to develop
  generative Deformable Models that take advantage of recent advances in
  component analysis and visual feature extraction in order to achieve accurate
  and robust performance without the need of large annotated training datasets.
  An additional aim is to compare the advantages and disadvantages between
  \emph{holistic} and \emph{part-based} appearance representations. A holistic
  appearance representation takes into account the texture that lies inside the
  whole surface of a deformable object. On the other hand, a part-based
  appearance representation extracts local texture patches that are centered 
  around the landmark points.

  \item \textbf{Objective 2: Propose methodologies for training Deformable
  Models with limited or even no human supervision} and explore solutions
  towards the online incremental update of these models with new training
  samples (\textbf{lifelong learning}). This refers to the task of
  constantly updating Deformable Models with images coming from the web $-$ in
  other words, the task of semi-automatic annotation of large
  collections of images. During the past twenty years, there has been huge
  dispute about whether generative or discriminative approaches are more
  appropriate for learning visual data~\cite{jordan2002discriminative}. Even
  though, there is no solid theoretical proof that discriminative models are
  always better than generative ones~\cite{jordan2002discriminative}, and in
  many cases the latter produce state-of-the-art
  results~\cite{antonakos2015feature,antonakos2014hog,tzimiropoulos2014gauss},
  the majority of researchers use discriminative models for learning from
  annotated data. However, discriminative methods are of limited use under an
  unsupervised setting. \emph{For the purpose of applications with minimal, or
  even no supervision, the family of generative techniques is more suitable.}
  Nevertheless, although the cost of manual annotation is well understood,
  unsupervised learning of Deformable Models has not received the proper
  attention and has been mainly restricted to controlled conditions and in small
  non-representative
  sets~\cite{baker2004automatic,kokkinos2007unsupervised,winn2005locus}.

  \item \textbf{Objective 3: Achieve state-of-the-art landmark localization
  performance by combining the advantages of generative and discriminative
  Deformable Models.} Discriminative (cascaded regression) Deformable Models
  have been shown to be more accurate and robust than generative models under
  challenging initializations. On the other had, generative models are very
  accurate when the initialization of their iterative optimization is
  reasonably close to the desired optimum solution. One of the objectives of
  this thesis is to analyze the main characteristics of these two families and
  create a unified model that benefits from their advantages and
  achieves state-of-the-art performance by \emph{outperforming both}.

  \item \textbf{Objective 4: Release an open-source implementation of all
  proposed approaches that contributes towards the need to standarize
  benchmarking.}
  One of the goals of this Ph.D. thesis is to accompany all the proposed
  methodologies of Objectives 1, 2 and 3 with a stable, tested and
  well-documented open-source implementation of both training and fitting. This
  can have a huge impact on the research community, since it allows to tweak
  with the proposed models and easily compare with them.
\end{itemize}
%%%%%%%%%%%%%

It should be highlighted that the ideas and methodologies presented in this
Ph.D. thesis are directly applicable to various deformable objects. However,
this work focuses entirely on the object of \textbf{human face}. The main
reasons behind that is that there are many large and carefully annotated
databases with facial images $-$ much more than for any other kind of
deformable object. In fact, academic research lacks annotated databases for the
vast majority of deformable objects. Furthermore, the human face is a very
representative example of an object that exhibits large variations in
deformations and appearance due to the plethora of facial expressions, race,
identity, gender, etc. In addition to that, it is an object of great
interest for many research fields with multiple applications. As a result,
almost all research on Deformable Models is applied and tested on the human
face. Recent large-scale challenges on facial
alignment~\cite{sagonas2013300,sagonas2013semi,sagonas2016faces} are characteristic examples of the rapid progress being made in the field.
