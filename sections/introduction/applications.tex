% !TEX root =  ../../thesis.tex
\section{Impact and Applications}
Generic Deformable Models that perform efficiently and accurately for a large
range of deformable objects have a tremendous impact on \emph{Human-Computer
Interaction} applications such as multi-modal interaction, entertainment,
digital arts, etc., and other fields like \emph{Robotics}, \emph{security},
etc. Furthermore, in the specific case of the human face, the task of facial
landmark localization is the cornerstone for various higher level applications
such as \emph{facial expressions recognition}, \emph{human behavior analysis},
\emph{face recognition/verification}, \emph{lip reading} and \emph{sign
language recognition}.

However, as mentioned before, one of the reasons that the task of landmark
localization has not advanced even more within the fields of Computer and Robot
Vision and has not expanded to more deformable objects is the cost of
annotations. This highlights the impact of developing unsupervised techniques
for learning Deformable Models which is immense, spanning a wide, diverse range
of applications, namely:
\begin{itemize}
  \item \emph{Consumer-level robots}, which would be able to learn ad-hoc
  detailed Deformable Models of various objects.

  \item The design of \emph{next generation Human-Computer Interaction} and
  \emph{Ubiquitous Computing} systems, assisting the rapidly growing area of
  first person vision systems.

  \item Paving the road for \emph{next generation Data Mining} and
  \emph{Information Retrieval} systems (\ie, analysis, indexing and retrieval
  of TV/Movie content in terms of actors appearance).
\end{itemize}

Additionally, the proposed ideas of this Ph.D. thesis, along with the provided
open-source implementations, have the potential to accelerate research in other
disciplines, such as \emph{Biology} and \emph{Psychology} and other life
sciences, by making the construction of complex detailed models of animals and
humans an affordable and easy - even for non computer scientists - task.

Finally, it should be noted that academic research suffers from lack of
annotated data for a large variety of objects. This fact highlights the
proposed ideas for learning Deformable Models with minimal annotation effort
can be a decisive step towards annotating large scale databases that can
greatly boost the research progress.
