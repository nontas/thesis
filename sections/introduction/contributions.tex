% !TEX root =  ../../thesis.tex
\section{Contributions}
In this section, the main contributions of this Ph.D. thesis are described in
more detail and related to the aforementioned objectives of
Sec.~\ref{sec:objectives}.

%%%%%%%%%%%%%%%%%
\begin{itemize}
  \item \textbf{Chapter~\ref{ch:aam}. Feature-based Active Appearance Models.}
  Lucas-Kanade (LK)~\cite{baker2003lucas,baker2004lucas} is a Gauss-Newton
  algorithm that has become the standard choice for performing parametric image
  alignment with respect to the parameters of an affine transform. Various
  alterations have been proposed depending on the characteristics of the
  performed optimization. Additionally, Active Appearance Models
  (AAMs)~\cite{cootes2001active,matthews2004active} is the most popular
  generative Deformable Model that employs the LK algorithm during fitting.
  Even though lots of improvements had been proposed for LK and AAMs, their
  performance was still poor compared to discriminative methodologies. In this
  chapter, we show that the combination of the non-linear least-squares
  optimization of a generative holistic Deformable Model with
  highly-descriptive, dense appearance features
  (\eg~HOG~\cite{dalal2005histograms}, SIFT~\cite{lowe1999object}) can achieve
  excellent performance for the task of face alignment. We show that even
  though the employment of dense features increases the data dimensionality,
  there is a small raise in the time complexity and a significant improvement
  in the alignment accuracy. The presented experiments also provide a
  comparison between various features and prove that HOG and SIFT are the most
  powerful. We present very accurate and robust experimental results for both
  face alignment and fitting with feature-based LK and holistic AAMs, that
  prove their invariance to illumination and expression changes and their
  generalization ability to unseen faces. Especially in the case of HOG and
  SIFT holistic AAMs, we demonstrate results on in-the-wild databases that
  significantly outperform various powerful and efficient discriminative
  Deformable Models. This chapter provides solution to Objective 1 in
  Sec.~\ref{sec:objectives}.

  \item \textbf{Chapter~\ref{ch:aps}. Active Pictorial Structures.} In this
  chapter, we exploit the effectiveness of part-based generative Deformable
  Models and shed light towards using a structure-based modeling for the shape
  and appearance of a deformable object. Specifically, we present a novel
  generative Deformable Model motivated by Pictorial Structures
  (PS)~\cite{fischler1973representation,felzenszwalb2005pictorial,andriluka2009pictorial} and
  AAMs~\cite{matthews2004active,antonakos2014hog,antonakos2015feature} for face
  alignment in-the-wild. Inspired by the tree structure used in PS, the
  proposed Active Pictorial Structures (APS) models the appearance of the
  object using multiple graph-based pairwise normal distributions (Gaussian
  Markov Random Field) between the patches extracted from the regions around
  adjacent landmarks. We show that this formulation is more accurate than using
  a single multivariate distribution (Principal Component Analysis) as commonly
  done in the literature. APS employs a weighted inverse compositional
  Gauss-Newton optimization with fixed Jacobian and Hessian that achieves close
  to real-time performance and state-of-the-art results. Finally, APS has a
  spring-like graph-based deformation prior term that makes them robust to bad
  initializations. We present extensive experiments on the task of face
  alignment, showing that APS outperforms many generative and discriminative
  Deformable Models. Note that APS is the first weighted inverse compositional
  technique that proves to be so accurate and efficient at the same time.
  Additionally, thanks to its formulation, APS is suitable for articulated
  deformable objects with multiple degrees of freedom, such as the human body,
  hand, etc. This chapter provides solution to Objective 1 of
  Sec.~\ref{sec:objectives}.

  \item \textbf{Chapter~\ref{ch:automatic_training}. Automatic Construction of
  Deformable Models.} As explained in Sec.~\ref{sec:problem_definition}, in
  order to train Deformable Models with good generalization ability, a large
  amount of carefully annotated data is required, which is a highly time
  consuming and costly task. In this chapter, we propose the first method for
  automatic construction of deformable models using images captured
  in-the-wild. The only requirements of the method are a crude bounding box
  object detector and a priori knowledge of the object’s shape (\eg~a point
  distribution model). The object detector can be as simple as the Viola-Jones
  algorithm~\cite{viola2001rapid,viola2004robust,viola2005detecting} (e.g. even
  the cheapest digital camera features a robust face detector). The 2D shape
  model can be created by using only a few shape examples with deformations. In
  our experiments on facial Deformable Models, we show that the proposed
  automatically built model not only performs well, but also outperforms
  discriminative models trained on carefully annotated data. Note that this
  chapter deals with Objective 2 in Sec.~\ref{sec:objectives} and the proposed
  methodology is the first one that shows that an automatically constructed
  model can perform as well as methods trained directly on annotated data.

  \item \textbf{Chapter~\ref{ch:acr}. Adaptive Cascaded Regression.} As
  explained in Sec.~\ref{sec:problem_definition}, the two predominant families
  of Deformable Models are: \emph{(i)}~discriminative models that employ
  cascaded
  regression~\cite{xiong2013supervised,ren2014face,kazemi2014one,asthana2014incremental,zhu2015face,tzimiropoulos2015project},
  and
  \emph{(ii)}~generative models optimized with the iterative Gauss-Newton
  algorithm~\cite{matthews2004active,papandreou2008adaptive,tzimiropoulos2012generic,antonakos2014hog,tzimiropoulos2014gauss,antonakos2015feature,alabort2016unified}.
  Although both of these approaches have been found to work well in practice,
  they each suffer from convergence issues. Cascaded regression has no
  theoretical guarantee of convergence to a local minimum and thus may fail to
  recover the fine details of the object. Gauss-Newton optimization is not
  robust to initializations that are far from the optimal solution. In this
  chapter, we propose to combine the best of these two worlds under a unified
  model, which directly answers Objective 3 in Sec.~\ref{sec:objectives}. We
  show that by combining the descent directions of cascaded regressors with the
  gradient descent directions from Gauss-Newton optimization, we can achieve
  both robustness to challenging initializations and accuracy with respect to
  fine details. Finally, we report state-of-the-art performance on the task of
  facial alignment against all current state-of-the-art generative and
  discriminative Deformable Models. Our experiments are shown on the latest and
  most challenging face alignment challenge and ACR is compared against
  methodologies that are trained on more data and are used by industrial
  companies.

  \item \textbf{Section~\ref{sec:menpo}. The Menpo Project.} An open-source
  implementation is provided for all the proposed methodologies within the
  Menpo Project~\cite{menpo2014,alabort2016menpo}. The Menpo Project is a set
  of open source, cross-platform Python frameworks and associated tooling that
  provide end-to-end solutions for 2D and 3D deformable modeling.This fulfills
  Objective 4 of Sec.~\ref{sec:objectives}.
\end{itemize}
%%%%%%%%%%%%%%%%%
