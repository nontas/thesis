% !TEX root =  ../../thesis.tex
\section{Facial Databases}
It should be highlighted that the ideas and methodologies presented in this Ph.D. thesis are directly applicable to various deformable objects. However, this work focuses entirely on the object of human face. The main reasons behind that are the following:
%%%%%%%%%%%%%%%
\begin{itemize}
  \item There are many large and carefully annotated databases with facial images $-$ much more than for any other kind of deformable object. In fact, academic research lacks annotated databases for the vast majority of deformable models.

  \item The human face is a very characteristic example of an object that exhibits large variations in deformations and appearance due to the plethora of facial expressions, range in race, identity, gender, etc. In addition to that, it is an object of great interest for many research fields and applications.
\end{itemize}
%%%%%%%%%%%%%%%

As a result, almost all research on Deformable Models during the last two decades is applied on the human face.


\begin{table}[!h]
\centering
\begin{tabular}{c|c||c}
\multicolumn{2}{c|}{\emph{Database}} & \emph{Number of images}\\
LFPW & trainset & 811\\
LFPW & testset & 224\\
HELEN & 2000\\
HELEN & 330\\
AFW & 337\\
IBUG & 135\\
300W & 600

\end{tabular}
\caption{Features parameters, neighbourhood size that contributes in each pixel's computation and number of channels.}
\label{tab:facial_databases}
\end{table}
