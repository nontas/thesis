% !TEX root =  ../../thesis.tex
\section{The Menpo Project}\label{sec:menpo}
An implementation of all the methodologies proposed in this Ph.D. thesis is
provided within the Menpo Project\footnote{The Menpo Project is an open-source
platform for all the stages of 2D and 3D Deformable Modeling. Website:
\url{http://www.menpo.org/}. Github:
\url{https://github.com/menpo/}}\textsuperscript{,}\footnote{The Menpo Project
is created and maintained by James Booth, Patrick Snape, Joan Alabort-i-Medina
and myself.}~\cite{menpo2014,alabort2016menpo}.
The Menpo Project is a set of open-source BSD licensed Python frameworks and
associated tooling that provide end-to-end solutions for 2D and 3D Deformable
Modeling. It aims to enable researchers, practitioners and students to easily
annotate new data sources and to investigate existing datasets. Of most
interest to the Computer Vision is the fact that the Menpo Project contains
completely open source implementations of a number of state-of-the-art
algorithms for face detection and deformable model building. Characteristic
examples of widely used state-of-the-art deformable model algorithms are Active
Appearance Models
(AAMs)~\cite{matthews2004active,antonakos2015feature,antonakos2014hog,tzimiropoulos2014gauss,tzimiropoulos2012generic,tzimiropoulos2014active,alabort2014bayesian},
Constrained Local Models~\cite{saragih2011deformable,asthana2015pixels} and
Supervised Descent Method~\cite{xiong2013supervised,asthana2014incremental}.

There is still a noteworthy lack of high quality open source software in the
field of Deformable Modeling. Most existing packages are encrypted, compiled,
non-maintained, partly documented, badly structured or difficult to modify.
This makes them unsuitable for adoption in cutting edge scientific research.
Consequently, research becomes even more difficult since performing a fair
comparison between existing methods is, in most cases, infeasible. For this
reason, the Menpo Project represents an important contribution towards open
science in the area. Additionally, it is important for Deformable Modeling to
move beyond the established area of facial annotations and to extend to a wide
variety of deformable object classes. Menpo can accelerate this progress by
providing all of our tools completely free and permissively licensed.

% Menpo logo
\begin{figure}[!t]
\centering
\includegraphics[width=0.156\linewidth]{figures/introduction/menpo/menpo}
\includegraphics[width=0.156\linewidth]{figures/introduction/menpo/menpofit}
\includegraphics[width=0.156\linewidth]{figures/introduction/menpo/menpodetect}
\includegraphics[width=0.156\linewidth]{figures/introduction/menpo/menpo3d}
\includegraphics[width=0.156\linewidth]{figures/introduction/menpo/menpowidgets}
\includegraphics[width=0.156\linewidth]{figures/introduction/menpo/menpocli}
\caption{The Menpo Project~\cite{menpo2014,alabort2016menpo} is an open-source platform that provides solutions for all the stages of 2D and 3D Deformable Modeling (\url{http://www.menpo.org/}). It includes implementations for all the methodologies proposed in this thesis.}
\label{fig:MenpoProject}
\end{figure}
%

The core functionality provided by the Menpo Project revolves around a powerful
and flexible cross-platform framework written in Python. This framework has a
number of subpackages, all of which rely on a core package called
\texttt{menpo}. The specialized subpackages are all based on top of
\texttt{menpo} and provide state-of-the-art Computer Vision algorithms in a
variety of areas (\texttt{menpofit}, \texttt{menpodetect}, \texttt{menpo3d},
\texttt{menpowidgets}).

\begin{itemize}
  \item \texttt{menpo}: This is a general purpose package that is designed from
  the ground up to make importing, manipulating and visualizing image and mesh
  data as simple as possible. In particular, we focus on data that has been
  annotated with a set of sparse landmarks. This form of data is common within
  the fields of Machine Learning and Computer Vision and is a prerequisite for
  constructing Deformable Models. All \texttt{menpo} core types are
  landmarkable and visualizing these landmarks is a primary concern of the
  \texttt{menpo} library. Since landmarks are first class citizens within
  \texttt{menpo}, it makes tasks like masking images, cropping images within
  the bounds of a set of landmarks, spatially transforming landmarks,
  extracting patches around landmarks and aligning images simple.

  \item \texttt{menpofit}: This package provides all the necessary tools for
  training and fitting a large variety of state-of-the-art Deformable Models
  under a unified framework, including the ones presented in this thesis. The
  provided methods are:
  %%%%%%%%%%%%%%%
  \begin{itemize}
    \item Active Appearance Model
    (AAM)~\cite{matthews2004active,antonakos2015feature,antonakos2014hog,tzimiropoulos2014gauss,tzimiropoulos2012generic,tzimiropoulos2014active,alabort2014bayesian}

    \item Supervised Descent Method
    (SDM)~\cite{xiong2013supervised,asthana2014incremental}

    \item Ensemble of Regression Trees (ERT) (powered by Dlib\footnote{Dlib
    Machine Learning toolkit:
    \url{http://dlib.net/}}~\cite{king2009dlib})~\cite{kazemi2014one}

    \item Constrained Local Model
    (CLM)~\cite{saragih2011deformable,asthana2015pixels}

    \item Active Shape Model (ASM)~\cite{cootes1995active}

    \item Active Pictorial Structures (APS)~\cite{antonakos2015active}

    \item Lucas-Kanade (LK) and Active Template Model
    (ATM)~\cite{baker2004lucas,baker2003lucas,lucey2013fourier,antonakos2015feature}
  \end{itemize}
  %%%%%%%%%%%%%%%

  \item \texttt{menpodetect}: This package contains methodologies for
  performing generic object detection in terms of a bounding box. The provided
  techniques include Viola-Jones object
  detector~\cite{viola2001rapid,viola2004robust,viola2005detecting,bradski2000opencv}, Support Vector Machines with HOG
  features~\cite{king2009dlib,king2015max}, Pico~\cite{markuvs2013object} and
  Deformable Part Model (DPM)~\cite{felzenszwalb2010object,mathias2014face}.

  \item \texttt{menpo3d}: It provides an open source implementation of 3D
  Morphable Models in-the-wild~\cite{blanz1999morphable}, as well as useful
  tools for importing, visualizing and transforming 3D data.

  \item \texttt{menpowidgets}: Package that includes widgets for ``fancy''
  visualization of \texttt{menpo} objects. It provides user friendly,
  aesthetically pleasing, interactive widgets for visualizing images, shapes,
  landmarks, trained models and fitting results.

  \item \texttt{menpocli}: Command Line Interface (CLI) for the Menpo Project
  that allows to readily use pre-trained state-of-the-art \texttt{menpofit}
  facial models.
\end{itemize}
%%%%%%%%%%%%%%%
