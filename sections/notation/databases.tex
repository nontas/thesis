% !TEX root =  ../../thesis.tex
\section{Facial Databases and Evaluation}\label{sec:notation:databases}
As explained in Chapter~\ref{ch:intro}, this thesis focuses on the
deformable object of human face. Specifically, we utilize all the commonly-used
in-the-wild databases that are annotated by
Sagonas~\emph{et al.}~\cite{sagonas2013semi,sagonas2013300,sagonas2016faces}
using the standard 68-point annotation mark-up proposed in the CMU MultiPIE
database~\cite{gross2010multi}. The employed in-the-wild databases, which
contain images downloaded from the web that are captured under totally
unconstrained conditions and exhibit large variations in pose, identity,
illumination, expressions, occlusion and resolution, include:
%%%%%%%%%%%%%
\begin{itemize}
  \item Labeled Face Parts in the Wild (LFPW)~\cite{belhumeur2011localizing} (811 training images, 224 testing images)
  \item Annotated Faces in the Wild (AFW)~\cite{zhu2012face} (337 images)
  \item HELEN~\cite{le2012interactive} (2000 training images, 330 testing images)
  \item IBUG~\cite{sagonas2013300,sagonas2013semi} (135 images)
  \item 300W~\cite{sagonas2013300,sagonas2013semi,sagonas2016faces} (600 images)
\end{itemize}
%%%%%%%%%%%%%
Note that we do not consider the original annotations of LFPW (29 points) or
HELEN (194 points), because recent
works~\cite{zhu2015face,zhang2016learning,ren2014face} have shown that
these databases have become saturated for the original annotations.
Figure~\ref{fig:databases} shows some examples from the employed in-the-wild
databases.

The fitting process is commonly initialized by computing the face's bounding box
using a face detector and then estimating the global similarity transform that
fits the mean shape within the bounding box boundaries. Note that this initial
similarity transform only involves a translation and scaling component and not
any in-plane rotation. The accuracy of a landmark localization result is
measured by the point-to-point RMS error between the fitted shape and the
ground-truth annotations, as proposed in~\cite{zhu2012face}. Denoting
$\mathbf{s}^f=[x_1^f, y_1^f, x_2^f, y_2^f, \ldots, x_n^f, y_n^f]^{\mathsf{T}}$
and
$\mathbf{s}^g=[x_1^g, y_1^g, x_2^g, y_2^g, \ldots, x_n^g, y_n^g]^{\mathsf{T}}$
as the fitted shape and the ground-truth shape, respectively, then the error
between them is expressed as
\begin{equation}
  \text{RMSE} = \frac{\sum_{i=1}^n \sqrt{(x_i^f - x_i^g)^2 + (y_i^f - y_i^g)^2}}{cn}
  \label{equ:error}
\end{equation}
where $c$ is a normalization constant. The interocular distance and the face
size, defined as
\begin{equation}
  c=\frac{(\max{\{x_i^g\}_1^n}-\min{\{x_i^g\}_1^n}+\max{\{y_i^g\}_1^n}-\min{\{y_i^g\}_1^n})}{2}
  \label{equ:error_normalization}
\end{equation}
are popular normalization choices.
These errors are presented in the form of Cumulative Error Distribution (CED)
and/or statistical measures.

%%%%%%%%%%%%%%
\begin{figure}[!h]
\centering
%%%%%%%%%
\subfloat[LFPW trainset]{\includegraphics[height=3.5cm]{figures/notation/databases/lfpw1}}
\hfill
\subfloat[LFPW testset]{\includegraphics[height=3.5cm]{figures/notation/databases/lfpw2}}
\hfill
\subfloat[HELEN trainset]{\includegraphics[height=3.5cm]{figures/notation/databases/helen1}}
\hfill
\subfloat[HELEN testset]{\includegraphics[height=3.5cm]{figures/notation/databases/helen2}}\\
\subfloat[AFW]{\includegraphics[height=3.5cm]{figures/notation/databases/afw}}
\hfill
\subfloat[IBUG]{\includegraphics[height=3.5cm]{figures/notation/databases/ibug}}
\hfill
\subfloat[300W Indoor]{\includegraphics[height=3.5cm]{figures/notation/databases/3001}}
\hfill
\subfloat[300W Outdoor]{\includegraphics[height=3.5cm]{figures/notation/databases/3002}}
%%%%%%%%%
\caption{Exemplar images from the employed in-the-wild databases.}
\label{fig:databases}
\end{figure}
%%%%%%%%%%%%%%



%%%%%%%%%%%%%%
\begin{figure}[!t]
\centering
%%%%%%%%%
\subfloat[$p_1=-3\sqrt{\lambda_1}$]{\includegraphics[width=0.2\linewidth]{figures/notation/appearance_model_gray_holistic/0_0}}
\subfloat[$p_1=-\frac{3}{2}\sqrt{\lambda_1}$]{\includegraphics[width=0.2\linewidth]{figures/notation/appearance_model_gray_holistic/0_1}}
\subfloat[$\bar{\mathbf{s}}$]{\includegraphics[width=0.2\linewidth]{figures/notation/appearance_model_gray_holistic/0_2}}
\subfloat[$p_1=\frac{3}{2}\sqrt{\lambda_1}$]{\includegraphics[width=0.2\linewidth]{figures/notation/appearance_model_gray_holistic/0_3}}
\subfloat[$p_1=3\sqrt{\lambda_1}$]{\includegraphics[width=0.2\linewidth]{figures/notation/appearance_model_gray_holistic/0_4}}\\
%%%%%%%%%
\subfloat[$p_2=-3\sqrt{\lambda_2}$]{\includegraphics[width=0.2\linewidth]{figures/notation/appearance_model_gray_holistic/1_0}}
\subfloat[$p_2=-\frac{3}{2}\sqrt{\lambda_2}$]{\includegraphics[width=0.2\linewidth]{figures/notation/appearance_model_gray_holistic/1_1}}
\subfloat[$\bar{\mathbf{s}}$]{\includegraphics[width=0.2\linewidth]{figures/notation/appearance_model_gray_holistic/1_2}}
\subfloat[$p_2=\frac{3}{2}\sqrt{\lambda_2}$]{\includegraphics[width=0.2\linewidth]{figures/notation/appearance_model_gray_holistic/1_3}}
\subfloat[$p_2=3\sqrt{\lambda_2}$]{\includegraphics[width=0.2\linewidth]{figures/notation/appearance_model_gray_holistic/1_4}}\\
%%%%%%%%%
\subfloat[$p_3=-3\sqrt{\lambda_3}$]{\includegraphics[width=0.2\linewidth]{figures/notation/appearance_model_gray_holistic/2_0}}
\subfloat[$p_3=-\frac{3}{2}\sqrt{\lambda_3}$]{\includegraphics[width=0.2\linewidth]{figures/notation/appearance_model_gray_holistic/2_1}}
\subfloat[$\bar{\mathbf{s}}$]{\includegraphics[width=0.2\linewidth]{figures/notation/appearance_model_gray_holistic/2_2}}
\subfloat[$p_3=\frac{3}{2}\sqrt{\lambda_3}$]{\includegraphics[width=0.2\linewidth]{figures/notation/appearance_model_gray_holistic/2_3}}
\subfloat[$p_3=3\sqrt{\lambda_3}$]{\includegraphics[width=0.2\linewidth]{figures/notation/appearance_model_gray_holistic/2_4}}\\
%%%%%%%%%
\subfloat[$p_4=-3\sqrt{\lambda_4}$]{\includegraphics[width=0.2\linewidth]{figures/notation/appearance_model_gray_holistic/3_0}}
\subfloat[$p_4=-\frac{3}{2}\sqrt{\lambda_4}$]{\includegraphics[width=0.2\linewidth]{figures/notation/appearance_model_gray_holistic/3_1}}
\subfloat[$\bar{\mathbf{s}}$]{\includegraphics[width=0.2\linewidth]{figures/notation/appearance_model_gray_holistic/3_2}}
\subfloat[$p_4=\frac{3}{2}\sqrt{\lambda_4}$]{\includegraphics[width=0.2\linewidth]{figures/notation/appearance_model_gray_holistic/3_3}}
\subfloat[$p_4=3\sqrt{\lambda_4}$]{\includegraphics[width=0.2\linewidth]{figures/notation/appearance_model_gray_holistic/3_4}}\\
%%%%%%%%%
\subfloat[$p_5=-3\sqrt{\lambda_5}$]{\includegraphics[width=0.2\linewidth]{figures/notation/appearance_model_gray_holistic/4_0}}
\subfloat[$p_5=-\frac{3}{2}\sqrt{\lambda_5}$]{\includegraphics[width=0.2\linewidth]{figures/notation/appearance_model_gray_holistic/4_1}}
\subfloat[$\bar{\mathbf{s}}$]{\includegraphics[width=0.2\linewidth]{figures/notation/appearance_model_gray_holistic/4_2}}
\subfloat[$p_5=\frac{3}{2}\sqrt{\lambda_5}$]{\includegraphics[width=0.2\linewidth]{figures/notation/appearance_model_gray_holistic/4_3}}
\subfloat[$p_5=3\sqrt{\lambda_5}$]{\includegraphics[width=0.2\linewidth]{figures/notation/appearance_model_gray_holistic/4_4}}
%%%%%%%%%
\caption{Exemplar instances of a holistic statistical appearance model trained
on the images of LFPW trainset. Each row shows the variations covered by the
first five principal components, where $\lambda_i$ is the eigenvalue that
corresponds to the $i$-th eigenvector.}
\label{fig:holistic_appearance_model_instances}
\end{figure}
%%%%%%%%%%%%%%


%%%%%%%%%%%%%%
\begin{figure}[!t]
\centering
%%%%%%%%%
\subfloat[$p_1=-3\sqrt{\lambda_1}$]{\includegraphics[width=0.2\linewidth]{figures/notation/appearance_model_gray_partbased/0_0}}
\subfloat[$p_1=-\frac{3}{2}\sqrt{\lambda_1}$]{\includegraphics[width=0.2\linewidth]{figures/notation/appearance_model_gray_partbased/0_1}}
\subfloat[$\bar{\mathbf{s}}$]{\includegraphics[width=0.2\linewidth]{figures/notation/appearance_model_gray_partbased/0_2}}
\subfloat[$p_1=\frac{3}{2}\sqrt{\lambda_1}$]{\includegraphics[width=0.2\linewidth]{figures/notation/appearance_model_gray_partbased/0_3}}
\subfloat[$p_1=3\sqrt{\lambda_1}$]{\includegraphics[width=0.2\linewidth]{figures/notation/appearance_model_gray_partbased/0_4}}\\
%%%%%%%%%
\subfloat[$p_2=-3\sqrt{\lambda_2}$]{\includegraphics[width=0.2\linewidth]{figures/notation/appearance_model_gray_partbased/1_0}}
\subfloat[$p_2=-\frac{3}{2}\sqrt{\lambda_2}$]{\includegraphics[width=0.2\linewidth]{figures/notation/appearance_model_gray_partbased/1_1}}
\subfloat[$\bar{\mathbf{s}}$]{\includegraphics[width=0.2\linewidth]{figures/notation/appearance_model_gray_partbased/1_2}}
\subfloat[$p_2=\frac{3}{2}\sqrt{\lambda_2}$]{\includegraphics[width=0.2\linewidth]{figures/notation/appearance_model_gray_partbased/1_3}}
\subfloat[$p_2=3\sqrt{\lambda_2}$]{\includegraphics[width=0.2\linewidth]{figures/notation/appearance_model_gray_partbased/1_4}}\\
%%%%%%%%%
\subfloat[$p_3=-3\sqrt{\lambda_3}$]{\includegraphics[width=0.2\linewidth]{figures/notation/appearance_model_gray_partbased/2_0}}
\subfloat[$p_3=-\frac{3}{2}\sqrt{\lambda_3}$]{\includegraphics[width=0.2\linewidth]{figures/notation/appearance_model_gray_partbased/2_1}}
\subfloat[$\bar{\mathbf{s}}$]{\includegraphics[width=0.2\linewidth]{figures/notation/appearance_model_gray_partbased/2_2}}
\subfloat[$p_3=\frac{3}{2}\sqrt{\lambda_3}$]{\includegraphics[width=0.2\linewidth]{figures/notation/appearance_model_gray_partbased/2_3}}
\subfloat[$p_3=3\sqrt{\lambda_3}$]{\includegraphics[width=0.2\linewidth]{figures/notation/appearance_model_gray_partbased/2_4}}\\
%%%%%%%%%
\subfloat[$p_4=-3\sqrt{\lambda_4}$]{\includegraphics[width=0.2\linewidth]{figures/notation/appearance_model_gray_partbased/3_0}}
\subfloat[$p_4=-\frac{3}{2}\sqrt{\lambda_4}$]{\includegraphics[width=0.2\linewidth]{figures/notation/appearance_model_gray_partbased/3_1}}
\subfloat[$\bar{\mathbf{s}}$]{\includegraphics[width=0.2\linewidth]{figures/notation/appearance_model_gray_partbased/3_2}}
\subfloat[$p_4=\frac{3}{2}\sqrt{\lambda_4}$]{\includegraphics[width=0.2\linewidth]{figures/notation/appearance_model_gray_partbased/3_3}}
\subfloat[$p_4=3\sqrt{\lambda_4}$]{\includegraphics[width=0.2\linewidth]{figures/notation/appearance_model_gray_partbased/3_4}}\\
%%%%%%%%%
\subfloat[$p_5=-3\sqrt{\lambda_5}$]{\includegraphics[width=0.2\linewidth]{figures/notation/appearance_model_gray_partbased/4_0}}
\subfloat[$p_5=-\frac{3}{2}\sqrt{\lambda_5}$]{\includegraphics[width=0.2\linewidth]{figures/notation/appearance_model_gray_partbased/4_1}}
\subfloat[$\bar{\mathbf{s}}$]{\includegraphics[width=0.2\linewidth]{figures/notation/appearance_model_gray_partbased/4_2}}
\subfloat[$p_5=\frac{3}{2}\sqrt{\lambda_5}$]{\includegraphics[width=0.2\linewidth]{figures/notation/appearance_model_gray_partbased/4_3}}
\subfloat[$p_5=3\sqrt{\lambda_5}$]{\includegraphics[width=0.2\linewidth]{figures/notation/appearance_model_gray_partbased/4_4}}
%%%%%%%%%
\caption{Exemplar instances of a part-based statistical appearance model
trained on the images of LFPW trainset. Each row shows the variations covered
by the first five principal components, where $\lambda_i$ is the eigenvalue
that corresponds to the $i$-th eigenvector.}
\label{fig:part_based_appearance_model_instances}
\end{figure}
%%%%%%%%%%%%%%
